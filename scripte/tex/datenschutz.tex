% vorlage-main.tex
\documentclass[12pt,a4paper]{scrartcl}
\usepackage{vorlage-design-main}% vorlage-design-main.sty
% Bildgröße global ändern
\setkeys{Gin}{width=0.75\linewidth}
% Literatur
\addbibresource{referenzen.bib}
% Anpassung des Quellcode-Stils
\lstset{
  basicstyle=\ttfamily\small,   % Ändern Sie die Schriftgröße des Quellcodes, falls erforderlich.
  columns=fullflexible,
  breaklines=true,              % Zeilenumbrüche für zu lange Zeilen
  postbreak=\mbox{$\hookrightarrow$\space}, % Pfeil am Zeilenumbruch
  literate={ö}{{\"o}}1
           {ä}{{\"a}}1
           {ü}{{\"u}}1
           {Ö}{{\"O}}1
           {Ä}{{\"A}}1
           {Ü}{{\"U}}1
           {ß}{{\ss}}1,
  xleftmargin=2em,              % Optional: Linker Abstand
  xrightmargin=3em,             % Optional: Rechter Abstand
  showstringspaces=false,
  showspaces=false             % Zeigt keine Leerzeichen an
}
% Hyperlinks
\hypersetup{
    colorlinks=true,
    linkcolor=blue,
    filecolor=magenta,
    urlcolor=cyan,
}
% Fehler wenn pandoc - Markdown in Latex
\newcommand{\tightlist}{
  \setlength{\itemsep}{0pt}\setlength{\parskip}{0pt}
}
% Titel, Autor und Datum
\title{Mein optimiertes Dokument}
%\author{Jan Unger}
\date{\today}

\begin{document}
\maketitle

\hypertarget{datenschutz}{%
\section{Datenschutz}\label{datenschutz}}

\hypertarget{datenschutz-auf-einen-blick}{%
\subsection{Datenschutz auf einen
Blick}\label{datenschutz-auf-einen-blick}}

\textbf{Allgemeine Hinweise}

Die folgenden Hinweise geben einen einfachen Überblick darüber, was mit
Ihren personenbezogenen Daten passiert, wenn Sie unsere Website
besuchen. Personenbezogene Daten sind alle Daten, mit denen Sie
persönlich identifiziert werden können. Ausführliche Informationen zum
Thema Datenschutz entnehmen Sie unserer unter diesem Text aufgeführten
Datenschutzerklärung.

\textbf{Datenerfassung auf unserer Website}

Wer ist verantwortlich für die Datenerfassung auf dieser Website?

Die Datenverarbeitung auf dieser Website erfolgt durch den
Websitebetreiber. Dessen Kontaktdaten können Sie dem Impressum dieser
Website entnehmen.

Wie erfassen wir Ihre Daten?

Ihre Daten werden zum einen dadurch erhoben, dass Sie uns diese
mitteilen. Hierbei kann es sich z.B. um Daten handeln, die Sie in ein
Kontaktformular eingeben.

Andere Daten werden automatisch beim Besuch der Website durch unsere
IT-Systeme erfasst. Das sind vor allem technische Daten (z.B.
Internetbrowser, Betriebssystem oder Uhrzeit des Seitenaufrufs). Die
Erfassung dieser Daten erfolgt automatisch, sobald Sie unsere Website
betreten.

Wofür nutzen wir Ihre Daten?

Ein Teil der Daten wird erhoben, um eine fehlerfreie Bereitstellung der
Website zu gewährleisten. Andere Daten können zur Analyse Ihres
Nutzerverhaltens verwendet werden.

Welche Rechte haben Sie bezüglich Ihrer Daten?

Sie haben jederzeit das Recht unentgeltlich Auskunft über Herkunft,
Empfänger und Zweck Ihrer gespeicherten personenbezogenen Daten zu
erhalten. Sie haben außerdem ein Recht, die Berichtigung, Sperrung oder
Löschung dieser Daten zu verlangen. Hierzu sowie zu weiteren Fragen zum
Thema Datenschutz können Sie sich jederzeit unter der im Impressum
angegebenen Adresse an uns wenden. Des Weiteren steht Ihnen ein
Beschwerderecht bei der zuständigen Aufsichtsbehörde zu.

\textbf{Analyse-Tools und Tools von Drittanbietern}

Beim Besuch unserer Website kann Ihr Surf-Verhalten statistisch
ausgewertet werden. Das geschieht vor allem mit Cookies und mit
sogenannten Analyseprogrammen. Die Analyse Ihres Surf-Verhaltens erfolgt
in der Regel anonym; das Surf-Verhalten kann nicht zu Ihnen
zurückverfolgt werden. Sie können dieser Analyse widersprechen oder sie
durch die Nichtbenutzung bestimmter Tools verhindern. Detaillierte
Informationen dazu finden Sie in der folgenden Datenschutzerklärung.

Sie können dieser Analyse widersprechen. Über die
Widerspruchsmöglichkeiten werden wir Sie in dieser Datenschutzerklärung
informieren.

\hypertarget{allgemeine-hinweise-und-pflichtinformationen}{%
\subsection{Allgemeine Hinweise und
Pflichtinformationen}\label{allgemeine-hinweise-und-pflichtinformationen}}

\textbf{Datenschutz}

Die Betreiber dieser Seiten nehmen den Schutz Ihrer persönlichen Daten
sehr ernst. Wir behandeln Ihre personenbezogenen Daten vertraulich und
entsprechend der gesetzlichen Datenschutzvorschriften sowie dieser
Datenschutzerklärung.

Wenn Sie diese Website benutzen, werden verschiedene personenbezogene
Daten erhoben. Personenbezogene Daten sind Daten, mit denen Sie
persönlich identifiziert werden können. Die vorliegende
Datenschutzerklärung erläutert, welche Daten wir erheben und wofür wir
sie nutzen. Sie erläutert auch, wie und zu welchem Zweck das geschieht.

Wir weisen darauf hin, dass die Datenübertragung im Internet (z.B. bei
der Kommunikation per E-Mail) Sicherheitslücken aufweisen kann. Ein
lückenloser Schutz der Daten vor dem Zugriff durch Dritte ist nicht
möglich.

\textbf{Hinweis zur verantwortlichen Stelle}

Die verantwortliche Stelle für die Datenverarbeitung auf dieser Website
ist:

Jan Unger Mackensenstrasse 18 42329 Wuppertal

E-Mail: esel573 (at) gmail (punkt) com

Verantwortliche Stelle ist die natürliche oder juristische Person, die
allein oder gemeinsam mit anderen über die Zwecke und Mittel der
Verarbeitung von personenbezogenen Daten (z.B. Namen, E-Mail-Adressen o.
Ä.) entscheidet.

\textbf{Widerruf Ihrer Einwilligung zur Datenverarbeitung}

Viele Datenverarbeitungsvorgänge sind nur mit Ihrer ausdrücklichen
Einwilligung möglich. Sie können eine bereits erteilte Einwilligung
jederzeit widerrufen. Dazu reicht eine formlose Mitteilung per E-Mail an
uns. Die Rechtmäßigkeit der bis zum Widerruf erfolgten Datenverarbeitung
bleibt vom Widerruf unberührt.

\textbf{Beschwerderecht bei der zuständigen Aufsichtsbehörde}

Im Falle datenschutzrechtlicher Verstöße steht dem Betroffenen ein
Beschwerderecht bei der zuständigen Aufsichtsbehörde zu. Zuständige
Aufsichtsbehörde in datenschutzrechtlichen Fragen ist der
Landesdatenschutzbeauftragte des Bundeslandes, in dem unser Unternehmen
seinen Sitz hat. Eine Liste der Datenschutzbeauftragten sowie deren
Kontaktdaten können folgendem Link entnommen werden:
\url{https://www.bfdi.bund.de/DE/Infothek/Anschriften_Links/anschriften_links-node.html}.

\textbf{Recht auf Datenübertragbarkeit}

Sie haben das Recht, Daten, die wir auf Grundlage Ihrer Einwilligung
oder in Erfüllung eines Vertrags automatisiert verarbeiten, an sich oder
an einen Dritten in einem gängigen, maschinenlesbaren Format aushändigen
zu lassen. Sofern Sie die direkte Übertragung der Daten an einen anderen
Verantwortlichen verlangen, erfolgt dies nur, soweit es technisch
machbar ist.

\textbf{SSL- bzw. TLS-Verschlüsselung}

Diese Seite nutzt aus Sicherheitsgründen und zum Schutz der Übertragung
vertraulicher Inhalte, wie zum Beispiel Bestellungen oder Anfragen, die
Sie an uns als Seitenbetreiber senden, eine SSL-bzw.
TLS-Verschlüsselung. Eine verschlüsselte Verbindung erkennen Sie daran,
dass die Adresszeile des Browsers von ``http://'' auf ``https://''
wechselt und an dem Schloss-Symbol in Ihrer Browserzeile.

Wenn die SSL- bzw. TLS-Verschlüsselung aktiviert ist, können die Daten,
die Sie an uns übermitteln, nicht von Dritten mitgelesen werden.

\textbf{Auskunft, Sperrung, Löschung}

Sie haben im Rahmen der geltenden gesetzlichen Bestimmungen jederzeit
das Recht auf unentgeltliche Auskunft über Ihre gespeicherten
personenbezogenen Daten, deren Herkunft und Empfänger und den Zweck der
Datenverarbeitung und ggf. ein Recht auf Berichtigung, Sperrung oder
Löschung dieser Daten. Hierzu sowie zu weiteren Fragen zum Thema
personenbezogene Daten können Sie sich jederzeit unter der im Impressum
angegebenen Adresse an uns wenden.

\textbf{Widerspruch gegen Werbe-Mails}

Der Nutzung von im Rahmen der Impressumspflicht veröffentlichten
Kontaktdaten zur Übersendung von nicht ausdrücklich angeforderter
Werbung und Informationsmaterialien wird hiermit widersprochen. Die
Betreiber der Seiten behalten sich ausdrücklich rechtliche Schritte im
Falle der unverlangten Zusendung von Werbeinformationen, etwa durch
Spam-E-Mails, vor.

\hypertarget{datenerfassung-auf-unserer-website}{%
\subsection{Datenerfassung auf unserer
Website}\label{datenerfassung-auf-unserer-website}}

\textbf{Cookies}

Die Internetseiten verwenden teilweise so genannte Cookies. Cookies
richten auf Ihrem Rechner keinen Schaden an und enthalten keine Viren.
Cookies dienen dazu, unser Angebot nutzerfreundlicher, effektiver und
sicherer zu machen. Cookies sind kleine Textdateien, die auf Ihrem
Rechner abgelegt werden und die Ihr Browser speichert.

Die meisten der von uns verwendeten Cookies sind so genannte
``Session-Cookies''. Sie werden nach Ende Ihres Besuchs automatisch
gelöscht. Andere Cookies bleiben auf Ihrem Endgerät gespeichert bis Sie
diese löschen. Diese Cookies ermöglichen es uns, Ihren Browser beim
nächsten Besuch wiederzuerkennen.

Sie können Ihren Browser so einstellen, dass Sie über das Setzen von
Cookies informiert werden und Cookies nur im Einzelfall erlauben, die
Annahme von Cookies für bestimmte Fälle oder generell ausschließen sowie
das automatische Löschen der Cookies beim Schließen des Browser
aktivieren. Bei der Deaktivierung von Cookies kann die Funktionalität
dieser Website eingeschränkt sein.

Cookies, die zur Durchführung des elektronischen Kommunikationsvorgangs
oder zur Bereitstellung bestimmter, von Ihnen erwünschter Funktionen
(z.B. Warenkorbfunktion) erforderlich sind, werden auf Grundlage von
Art. 6 Abs. 1 lit. f DSGVO gespeichert. Der Websitebetreiber hat ein
berechtigtes Interesse an der Speicherung von Cookies zur technisch
fehlerfreien und optimierten Bereitstellung seiner Dienste. Soweit
andere Cookies (z.B. Cookies zur Analyse Ihres Surfverhaltens)
gespeichert werden, werden diese in dieser Datenschutzerklärung
gesondert behandelt.

\textbf{Server-Log-Dateien}

Der Provider der Seiten erhebt und speichert automatisch Informationen
in so genannten Server-Log-Dateien, die Ihr Browser automatisch an uns
übermittelt. Dies sind:

\begin{itemize}
\tightlist
\item
  Browsertyp und Browserversion
\item
  verwendetes Betriebssystem
\item
  Referrer URL
\item
  Hostname des zugreifenden Rechners
\item
  Uhrzeit der Serveranfrage
\item
  IP-Adresse
\end{itemize}

Eine Zusammenführung dieser Daten mit anderen Datenquellen wird nicht
vorgenommen.

Grundlage für die Datenverarbeitung ist Art. 6 Abs. 1 lit. b DSGVO, der
die Verarbeitung von Daten zur Erfüllung eines Vertrags oder
vorvertraglicher Maßnahmen gestattet.

\textbf{Kontaktformular}

Wenn Sie uns per Kontaktformular Anfragen zukommen lassen, werden Ihre
Angaben aus dem Anfrageformular inklusive der von Ihnen dort angegebenen
Kontaktdaten zwecks Bearbeitung der Anfrage und für den Fall von
Anschlussfragen bei uns gespeichert. Diese Daten geben wir nicht ohne
Ihre Einwilligung weiter.

Die Verarbeitung der in das Kontaktformular eingegebenen Daten erfolgt
somit ausschließlich auf Grundlage Ihrer Einwilligung (Art. 6 Abs. 1
lit. a DSGVO). Sie können diese Einwilligung jederzeit widerrufen. Dazu
reicht eine formlose Mitteilung per E-Mail an uns. Die Rechtmäßigkeit
der bis zum Widerruf erfolgten Datenverarbeitungsvorgänge bleibt vom
Widerruf unberührt.

Die von Ihnen im Kontaktformular eingegebenen Daten verbleiben bei uns,
bis Sie uns zur Löschung auffordern, Ihre Einwilligung zur Speicherung
widerrufen oder der Zweck für die Datenspeicherung entfällt (z.B. nach
abgeschlossener Bearbeitung Ihrer Anfrage). Zwingende gesetzliche
Bestimmungen -- insbesondere Aufbewahrungsfristen -- bleiben unberührt.

\hypertarget{analyse-tools-und-werbung}{%
\subsection{Analyse Tools und Werbung}\label{analyse-tools-und-werbung}}

\textbf{Google Analytics}

Diese Website nutzt Funktionen des Webanalysedienstes Google Analytics.
Anbieter ist die Google Inc., 1600 Amphitheatre Parkway, Mountain View,
CA 94043, USA.

Google Analytics verwendet so genannte ``Cookies''. Das sind
Textdateien, die auf Ihrem Computer gespeichert werden und die eine
Analyse der Benutzung der Website durch Sie ermöglichen. Die durch den
Cookie erzeugten Informationen über Ihre Benutzung dieser Website werden
in der Regel an einen Server von Google in den USA übertragen und dort
gespeichert.

Die Speicherung von Google-Analytics-Cookies erfolgt auf Grundlage von
Art. 6 Abs. 1 lit. f DSGVO. Der Websitebetreiber hat ein berechtigtes
Interesse an der Analyse des Nutzerverhaltens, um sowohl sein Webangebot
als auch seine Werbung zu optimieren.

\textbf{Browser Plugin}

Sie können die Speicherung der Cookies durch eine entsprechende
Einstellung Ihrer Browser-Software verhindern; wir weisen Sie jedoch
darauf hin, dass Sie in diesem Fall gegebenenfalls nicht sämtliche
Funktionen dieser Website vollumfänglich werden nutzen können. Sie
können darüber hinaus die Erfassung der durch den Cookie erzeugten und
auf Ihre Nutzung der Website bezogenen Daten (inkl. Ihrer IP-Adresse) an
Google sowie die Verarbeitung dieser Daten durch Google verhindern,
indem Sie das unter dem folgenden Link verfügbare Browser-Plugin
herunterladen und installieren:
\url{https://tools.google.com/dlpage/gaoptout?hl=de}.

\textbf{Widerspruch gegen Datenerfassung}

Sie können die Erfassung Ihrer Daten durch Google Analytics verhindern,
indem Sie auf folgenden Link klicken. Es wird ein Opt-Out-Cookie
gesetzt, der die Erfassung Ihrer Daten bei zukünftigen Besuchen dieser
Website verhindert: Google Analytics deaktivieren.

Mehr Informationen zum Umgang mit Nutzerdaten bei Google Analytics
finden Sie in der Datenschutzerklärung von Google:
\url{https://support.google.com/analytics/answer/6004245?hl=de}.

\textbf{Google reCAPTCHA}

Wir nutzen ``Google reCAPTCHA'' (im Folgenden ``reCAPTCHA'') auf unseren
Websites. Anbieter ist die Google Inc., 1600 Amphitheatre Parkway,
Mountain View, CA 94043, USA (``Google'').

Mit reCAPTCHA soll überprüft werden, ob die Dateneingabe auf unseren
Websites (z.B. in einem Kontaktformular) durch einen Menschen oder durch
ein automatisiertes Programm erfolgt. Hierzu analysiert reCAPTCHA das
Verhalten des Websitebesuchers anhand verschiedener Merkmale. Diese
Analyse beginnt automatisch, sobald der Websitebesucher die Website
betritt. Zur Analyse wertet reCAPTCHA verschiedene Informationen aus
(z.B. IP-Adresse, Verweildauer des Websitebesuchers auf der Website oder
vom Nutzer getätigte Mausbewegungen). Die bei der Analyse erfassten
Daten werden an Google weitergeleitet.

Die reCAPTCHA-Analysen laufen vollständig im Hintergrund.
Websitebesucher werden nicht darauf hingewiesen, dass eine Analyse
stattfindet.

Die Datenverarbeitung erfolgt auf Grundlage von Art. 6 Abs. 1 lit. f
DSGVO. Der Websitebetreiber hat ein berechtigtes Interesse daran, seine
Webangebote vor missbräuchlicher automatisierter Ausspähung und vor SPAM
zu schützen.

Weitere Informationen zu Google reCAPTCHA sowie die Datenschutzerklärung
von Google entnehmen Sie folgenden Links:
\url{https://www.google.com/intl/de/policies/privacy/} und
\url{https://www.google.com/recaptcha/intro/android.html}.

\hypertarget{plugins-und-tools}{%
\subsection{Plugins und Tools}\label{plugins-und-tools}}

\textbf{Google Web Fonts}

Diese Seite nutzt zur einheitlichen Darstellung von Schriftarten so
genannte Web Fonts, die von Google bereitgestellt werden. Beim Aufruf
einer Seite lädt Ihr Browser die benötigten Web Fonts in ihren
Browsercache, um Texte und Schriftarten korrekt anzuzeigen.

Zu diesem Zweck muss der von Ihnen verwendete Browser Verbindung zu den
Servern von Google aufnehmen. Hierdurch erlangt Google Kenntnis darüber,
dass über Ihre IP-Adresse unsere Website aufgerufen wurde. Die Nutzung
von Google Web Fonts erfolgt im Interesse einer einheitlichen und
ansprechenden Darstellung unserer Online-Angebote. Dies stellt ein
berechtigtes Interesse im Sinne von Art. 6 Abs. 1 lit. f DSGVO dar.

Wenn Ihr Browser Web Fonts nicht unterstützt, wird eine Standardschrift
von Ihrem Computer genutzt.

Weitere Informationen zu Google Web Fonts finden Sie unter
\url{https://developers.google.com/fonts/faq} und in der
Datenschutzerklärung von Google:
\url{https://www.google.com/policies/privacy/}.

\textbf{Google Maps}

Diese Seite nutzt über eine API den Kartendienst Google Maps. Anbieter
ist die Google Inc., 1600 Amphitheatre Parkway, Mountain View, CA 94043,
USA.

Zur Nutzung der Funktionen von Google Maps ist es notwendig, Ihre IP
Adresse zu speichern. Diese Informationen werden in der Regel an einen
Server von Google in den USA übertragen und dort gespeichert. Der
Anbieter dieser Seite hat keinen Einfluss auf diese Datenübertragung.

Die Nutzung von Google Maps erfolgt im Interesse einer ansprechenden
Darstellung unserer Online-Angebote und an einer leichten Auffindbarkeit
der von uns auf der Website angegebenen Orte. Dies stellt ein
berechtigtes Interesse im Sinne von Art. 6 Abs. 1 lit. f DSGVO dar.

Mehr Informationen zum Umgang mit Nutzerdaten finden Sie in der
Datenschutzerklärung von Google:
\url{https://www.google.de/intl/de/policies/privacy/}. % Platzhalter

\end{document}
