% vorlage-main.tex
\documentclass[12pt,a4paper]{scrartcl}
\usepackage{vorlage-design-main}% vorlage-design-main.sty
% Bildgröße global ändern
\setkeys{Gin}{width=0.75\linewidth}
% Literatur
\addbibresource{referenzen.bib}
% Anpassung des Quellcode-Stils
\lstset{
  basicstyle=\ttfamily\small,   % Ändern Sie die Schriftgröße des Quellcodes, falls erforderlich.
  columns=fullflexible,
  breaklines=true,              % Zeilenumbrüche für zu lange Zeilen
  postbreak=\mbox{$\hookrightarrow$\space}, % Pfeil am Zeilenumbruch
  literate={ö}{{\"o}}1
           {ä}{{\"a}}1
           {ü}{{\"u}}1
           {Ö}{{\"O}}1
           {Ä}{{\"A}}1
           {Ü}{{\"U}}1
           {ß}{{\ss}}1,
  xleftmargin=2em,              % Optional: Linker Abstand
  xrightmargin=3em,             % Optional: Rechter Abstand
  showstringspaces=false,
  showspaces=false             % Zeigt keine Leerzeichen an
}
% Hyperlinks
\hypersetup{
    colorlinks=true,
    linkcolor=blue,
    filecolor=magenta,
    urlcolor=cyan,
}
% Fehler wenn pandoc - Markdown in Latex
\newcommand{\tightlist}{
  \setlength{\itemsep}{0pt}\setlength{\parskip}{0pt}
}
% Titel, Autor und Datum
\title{Mein optimiertes Dokument}
%\author{Jan Unger}
\date{\today}

\begin{document}
\maketitle

\hypertarget{sourcecode-ein-projekt-blog-fuxfcr-technikenthusiasten}{%
\subsection{SourceCode -- Ein Projekt-Blog für
Technikenthusiasten}\label{sourcecode-ein-projekt-blog-fuxfcr-technikenthusiasten}}

Im digitalen Zeitalter sind Technologie und Programmierung nicht nur für
IT-Profis von Bedeutung. Der Blog ``SourceCode'' öffnet für alle
Technikbegeisterten die Tür zu einer Welt, in der Code und Kreativität
aufeinandertreffen. Es handelt sich nicht um einen gewöhnlichen Blog,
sondern um eine Plattform, die sich tiefgehend mit Themen wie Scripten,
Arduino, Mikrocontrollern und diversen Programmiersprachen wie Python
und C auseinandersetzt.

Doch nicht nur Softwarethemen stehen im Mittelpunkt. ``SourceCode''
taucht ebenfalls in die Welt der Vektorgrafiken ein und bietet
interessante Artikel zum Thema SVG, einem Standardformat für
zweidimensionale Vektorgrafiken. Das bedeutet, dass Designer und
Programmierer gleichermaßen wertvolle Informationen finden. Neben diesen
technisch orientierten Themen bietet der Blog auch Einblicke in den
Hindernislauf aus sportlicher Perspektive. Dieser Sport stellt eine
physische und mentale Herausforderung dar.

Insgesamt vermittelt ``SourceCode'' eine Kombination aus technischem
Know-how und der Leidenschaft für kontinuierliche Weiterbildung. In
einer sich ständig weiterentwickelnden Welt der
Technologie. % Platzhalter

\end{document}
