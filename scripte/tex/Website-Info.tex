% vorlage-main.tex
\documentclass[12pt,a4paper]{scrartcl}
\usepackage{vorlage-design-main}% vorlage-design-main.sty
% Bildgröße global ändern
\setkeys{Gin}{width=0.75\linewidth}
% Literatur
\addbibresource{referenzen.bib}
% Anpassung des Quellcode-Stils
\lstset{
  basicstyle=\ttfamily\small,   % Ändern Sie die Schriftgröße des Quellcodes, falls erforderlich.
  columns=fullflexible,
  breaklines=true,              % Zeilenumbrüche für zu lange Zeilen
  postbreak=\mbox{$\hookrightarrow$\space}, % Pfeil am Zeilenumbruch
  literate={ö}{{\"o}}1
           {ä}{{\"a}}1
           {ü}{{\"u}}1
           {Ö}{{\"O}}1
           {Ä}{{\"A}}1
           {Ü}{{\"U}}1
           {ß}{{\ss}}1,
  xleftmargin=2em,              % Optional: Linker Abstand
  xrightmargin=3em,             % Optional: Rechter Abstand
  showstringspaces=false,
  showspaces=false             % Zeigt keine Leerzeichen an
}
% Hyperlinks
\hypersetup{
    colorlinks=true,
    linkcolor=blue,
    filecolor=magenta,
    urlcolor=cyan,
}
% Fehler wenn pandoc - Markdown in Latex
\newcommand{\tightlist}{
  \setlength{\itemsep}{0pt}\setlength{\parskip}{0pt}
}
% Titel, Autor und Datum
\title{Mein optimiertes Dokument}
%\author{Jan Unger}
\date{\today}

\begin{document}
\maketitle

\hypertarget{website-info}{%
\section{Website-Info}\label{website-info}}

ChatGPT \url{https://chat.openai.com/}

\hypertarget{erstelle-und-strukturiere-mir-webseiteninhalte}{%
\subsection{Erstelle und Strukturiere mir
Webseiteninhalte}\label{erstelle-und-strukturiere-mir-webseiteninhalte}}

\textbf{Aufgabenstellung 1: Erstelle und Strukturiere mir
Webseiteninhalte}

Ziel dieser Aufgabe ist die Erstellung und Strukturierung von Dateien
für eine Website gemäß den angegebenen Spezifikationen.

\textbf{Anforderungen:}

\begin{enumerate}
\def\labelenumi{\arabic{enumi}.}
\tightlist
\item
  \textbf{Root-Ordner (Hauptverzeichnis):}

  \begin{itemize}
  \tightlist
  \item
    Erstellen Sie eine {\lstinline!index.html!} im
    Root-Ordner der Webseite.
  \end{itemize}
\item
  \textbf{CSS-Ordner:}

  \begin{itemize}
  \tightlist
  \item
    Erstellen Sie eine CSS-Datei im Ordner
    {\lstinline!css!}. Die genaue Bezeichnung der Datei kann
    frei gewählt werden, sollte jedoch beschreibend sein, z.B.
    {\lstinline!main.css!}.
  \end{itemize}
\item
  \textbf{Bilder-Ordner:}

  \begin{itemize}
  \tightlist
  \item
    Platzieren Sie einen Bildplatzhalter im Ordner
    {\lstinline!img!}. Dies kann ein Standardbild sein, das
    später durch ein tatsächliches Bild ersetzt wird.
  \end{itemize}
\item
  \textbf{PHP-Ordner:}

  \begin{itemize}
  \tightlist
  \item
    Erstellen Sie eine PHP-Datei im Ordner
    {\lstinline!php!}. Der Name der Datei sollte den Inhalt
    oder den Zweck der Datei widerspiegeln.
  \end{itemize}
\item
  \textbf{HTML-Ordner:}

  \begin{itemize}
  \tightlist
  \item
    Erstellen Sie eine zusätzliche HTML-Datei im Ordner
    {\lstinline!html!}. Die Bezeichnung dieser Datei kann
    frei gewählt werden, sollte jedoch den Inhalt oder Zweck der Datei
    deutlich machen.
  \end{itemize}
\end{enumerate}

\textbf{Hinweise:} - Achten Sie auf eine klare und saubere
Strukturierung der Dateien und Ordner. - Beachten Sie die Best Practices
bei der Benennung von Dateien und Ordnern, um Verwirrung oder Konflikte
zu vermeiden.

\textbf{Abgabe:} - Ein Zip-Archiv oder einen Link zu einem Repository,
das alle erstellten Dateien und Ordner enthält. - Ein kurzer Bericht
(optional), in dem die Struktur und der Zweck jeder Datei beschrieben
wird.

\textbf{Deadline:} - Abgabedatum: 1.9.2023

\textbf{Bewertungskriterien:} - die Qualität des Codes, die Einhaltung
der Best Practices und die Ästhetik des Designs.

\hypertarget{uxfcberpruxfcfe-und-optimiere-die-webseiteninhalte}{%
\subsubsection{Überprüfe und Optimiere die
Webseiteninhalte}\label{uxfcberpruxfcfe-und-optimiere-die-webseiteninhalte}}

\textbf{Aufgabenstellung 2: Überprüfe und Optimiere die
Webseiteninhalte}

Ziel dieser Aufgabe ist die Analyse und Optimierung von Webinhalten, um
eine bestmögliche Benutzererfahrung über verschiedene Geräte hinweg
sicherzustellen.

\textbf{Anforderungen:}

\begin{enumerate}
\def\labelenumi{\arabic{enumi}.}
\tightlist
\item
  \textbf{Best Practices für html, CSS und php:}

  \begin{itemize}
  \tightlist
  \item
    Untersuchen Sie das aktuelle CSS und vergewissern Sie sich, dass es
    den gängigen Best Practices entspricht.
  \end{itemize}
\item
  \textbf{Barrierefreiheit:}

  \begin{itemize}
  \tightlist
  \item
    Stellen Sie sicher, dass die Webseite für alle Benutzer zugänglich
    ist, einschließlich Menschen mit Behinderungen.
  \end{itemize}
\item
  \textbf{Responsive Design:}

  \begin{itemize}
  \tightlist
  \item
    Die Webseite sollte auf unterschiedlichen Geräten - von
    Mobiltelefonen über Tablets bis hin zu Desktop-Computern - gut
    lesbar und nutzbar sein.
  \end{itemize}
\item
  \textbf{Einheitlichkeit der Maßeinheiten:}

  \begin{itemize}
  \tightlist
  \item
    Verwenden Sie, wo immer möglich, die Maßeinheit ``em'' in Ihrem CSS,
    um Skalierbarkeit und Flexibilität über verschiedene Geräte und
    Bildschirmgrößen hinweg sicherzustellen.
  \end{itemize}
\item
  \textbf{Ziel-Dateien:}

  \begin{itemize}
  \tightlist
  \item
    Die Überprüfung und Anpassungen sollten sich auf ``Index.html''
    sowie die zugehörigen PHP-Seiten erstrecken.
  \end{itemize}
\end{enumerate}

\textbf{Abgabe:} - Überarbeitetes CSS-Dokument - Überarbeitete
index.html und PHP-Dateien, falls notwendig - Ein kurzer Bericht über
die vorgenommenen Änderungen und deren Auswirkungen auf die
Benutzererfahrung.

\hypertarget{kontrolliere-nochmals}{%
\subsubsection{Kontrolliere nochmals}\label{kontrolliere-nochmals}}

\begin{lstlisting}[language=bash]
# ChatGPT
Nehmen Sie die Rolle eines erfahrenen Webprogrammierer und Webdesigner ein.

Aufgabe: Kontrolliere nochmals unter Berücksichtigung von

- Best Practices  
- Barrierefreiheit
- Responsive Design: gute Lesbarkeit auf Handy, Tablett und große Bildschirme
- Text soll vertikal zentriert und im Blickfeld des Betrachters sein
- Maßeinheiten in em
- Native Lazy Loading (für moderne Browser)
- Performance

Datei = " "

für index.html, html und php, css
\end{lstlisting}

\hypertarget{css-optimieren}{%
\subsubsection{CSS optimieren}\label{css-optimieren}}

\begin{lstlisting}[language=bash]
# ChatGPT
optimiere meine CSS und Vermeide teure CSS-Selektoren 

- Zusammenfassung von Wiederholungen
- Vermeidung von spezifischen Kombinationen:
- Optimierung der Media Queries: 

datei = " "
\end{lstlisting}

\hypertarget{schreibstile}{%
\subsection{Schreibstile}\label{schreibstile}}

\begin{lstlisting}[language=bash]
# ChatGPT
# Favorit: Expositorisch
# Schreibstile: Beschreibend, Expositorisch, Reflektierend, Akademisch, Kritisch
Schreibstil: Expositorisch
Erstellen Sie je nach Schreibstil eine kurze (ca. 35 Wörter) und ansprechende Zusammenfassung des folgenden Artikels. Die Zusammenfassung sollte für jemanden ohne wissenschaftlichen Hintergrund verständlich sein und gleichzeitig die wichtigsten Informationen genau wiedergeben. 
Artikel: " "
\end{lstlisting}

\hypertarget{html-grundgeruxfcsts}{%
\subsection{HTML-Grundgerüsts}\label{html-grundgeruxfcsts}}

\textbf{1. Best Practices:}

\begin{itemize}
\tightlist
\item
  Das DOCTYPE-Element ist korrekt am Anfang des Dokuments angegeben.
\item
  Das {\lstinline!lang!}-Attribut im
  {\lstinline!<html>!}-Tag ist gesetzt und zeigt an, dass
  der Inhalt auf Deutsch ist.
\item
  Es gibt eine klare Trennung von Struktur (HTML) und Stil (CSS).
\item
  Es wird ein verständlicher und barrierefreier Alternativtext für
  Bilder verwendet.
\end{itemize}

\textbf{2. Barrierefreiheit:}

\begin{itemize}
\tightlist
\item
  Semantische HTML-Tags wie {\lstinline!<header>!},
  {\lstinline!<nav>!}, {\lstinline!<main>!} und
  {\lstinline!<footer>!} werden verwendet, um den Inhalt zu
  strukturieren.
\item
  Es werden Rollen wie {\lstinline!role="banner"!},
  {\lstinline!role="navigation"!},
  {\lstinline!role="main"!}, und
  {\lstinline!role="contentinfo"!} verwendet, um die
  Semantik weiter zu erhöhen.
\item
  Es gibt einen Alternativtext für das Bild, der seine Bedeutung oder
  Funktion beschreibt.
\end{itemize}

\textbf{3. Responsive Design:}

\begin{itemize}
\tightlist
\item
  Der Meta-Viewport-Tag
  {\lstinline!<meta name="viewport" content="width=device-width, initial-scale=1.0">!}
  ist vorhanden und sorgt dafür, dass die Seite korrekt auf
  verschiedenen Bildschirmgrößen skaliert.
\item
  Für eine vollständige Überprüfung des responsiven Designs müssten wir
  den CSS-Code betrachten und die Webseite in verschiedenen
  Bildschirmgrößen testen. Es ist wichtig, dass das Layout der Webseite
  sich anpasst und der Inhalt auf verschiedenen Geräten gut lesbar
  bleibt.
\end{itemize}

\textbf{4. Maßeinheiten in {\lstinline!em!}:}

\begin{itemize}
\tightlist
\item
  Der bereitgestellte HTML-Code verwendet keine direkten Stilangaben,
  daher müssen alle Größen- und Abstandsangaben im CSS in
  {\lstinline!em!}-Einheiten gemacht werden.
\end{itemize}

\textbf{Was noch zu tun ist:}

\begin{itemize}
\tightlist
\item
  Testen Sie die Webseite mit einem Screenreader oder
  Barrierefreiheits-Tool, um sicherzustellen, dass sie für alle Benutzer
  zugänglich ist.
\item
  Überprüfen Sie alle anderen HTML- und PHP-Seiten mit den gleichen
  Kriterien.
\item
  Stellen Sie sicher, dass alle externen Ressourcen (Bilder, Skripte,
  Stylesheets) korrekt geladen werden und keine gebrochenen Links
  vorhanden sind.
\item
  Das CSS sollte regelmäßig überprüft werden, um sicherzustellen, dass
  es den Best Practices entspricht und effizient ist.
\end{itemize}

Es wäre auch hilfreich, regelmäßige Tests mit echten Benutzern
durchzuführen, um Feedback zu Design, Usability und Barrierefreiheit zu
erhalten. Dies wird Ihnen helfen, Bereiche zu identifizieren, die weiter
optimiert werden müssen.

\hypertarget{performance-einer-website-zu-verbessern}{%
\subsection{Performance einer Website zu
verbessern}\label{performance-einer-website-zu-verbessern}}

Die Performance einer Website zu verbessern, kann das Benutzererlebnis
erheblich steigern und sich positiv auf die SEO auswirken.

\begin{enumerate}
\def\labelenumi{\arabic{enumi}.}
\tightlist
\item
  \textbf{Bilder optimieren:}

  \begin{itemize}
  \tightlist
  \item
    Verwende Formate wie WebP, das oft kleinere Dateigrößen als PNG oder
    JPEG bietet.
  \item
    Komprimiere Bilder, ohne die Qualität wesentlich zu beeinträchtigen.
  \item
    Nutze Responsive Images mit dem {\lstinline!srcset!}
    Attribut, um verschiedene Bildgrößen für verschiedene
    Bildschirmauflösungen bereitzustellen.
  \end{itemize}
\item
  \textbf{Minimieren und kombinieren von Dateien:}

  \begin{itemize}
  \tightlist
  \item
    Minimiere CSS, JavaScript und HTML.
  \item
    Kombiniere mehrere CSS- oder JS-Dateien in eine einzelne Datei, um
    HTTP-Anfragen zu reduzieren.
  \end{itemize}
\item
  \textbf{Verwende Lazy Loading:}

  \begin{itemize}
  \tightlist
  \item
    Lade Bilder, Videos und Iframes erst, wenn sie im Viewport des
    Benutzers erscheinen.
  \end{itemize}
\item
  \textbf{Browser-Caching nutzen:}

  \begin{itemize}
  \tightlist
  \item
    Durch das Festlegen von Ablaufzeiten für Ressourcen können
    wiederkehrende Besucher Inhalte aus ihrem Browser-Cache laden.
  \end{itemize}
\item
  \textbf{Verwende Content Delivery Networks (CDNs):}

  \begin{itemize}
  \tightlist
  \item
    Ein CDN kann Inhalte schneller an Benutzer weltweit liefern, indem
    es Kopien deiner Ressourcen auf mehreren Servern verteilt.
  \end{itemize}
\item
  \textbf{Optimiere die Reihenfolge der Ressourcen:}

  \begin{itemize}
  \tightlist
  \item
    Lade kritischen CSS und JS zuerst und verschiebe nicht-kritische
    JS-Ressourcen mit dem {\lstinline!defer!} Attribut.
  \end{itemize}
\item
  \textbf{Verringere Serverantwortzeit:}

  \begin{itemize}
  \tightlist
  \item
    Betrachte Hosting-Optionen und optimiere Backend-Code. Ein
    effizientes Webhosting und ein optimierter Server können die
    Antwortzeit verbessern.
  \end{itemize}
\item
  \textbf{Verwende HTTP/3 oder HTTP/2:}

  \begin{itemize}
  \tightlist
  \item
    Diese neueren Protokolle bieten Verbesserungen gegenüber HTTP/1.1,
    wie z.B. das Multiplexing von Anfragen.
  \end{itemize}
\item
  \textbf{Reduziere externe Ressourcen und Plugins:}

  \begin{itemize}
  \tightlist
  \item
    Jede externe Ressource (z.B. Schriftarten, Widgets, Plugins) kann
    die Ladezeit erhöhen.
  \end{itemize}
\item
  \textbf{Optimiere CSS und JavaScript:}

  \begin{itemize}
  \tightlist
  \item
    Vermeide teure CSS-Selektoren und reduziere die Tiefe von
    JavaScript-Aufrufen.
  \end{itemize}
\item
  \textbf{Nutze serverseitige Kompression:}

  \begin{itemize}
  \tightlist
  \item
    Verwende Gzip oder Brotli, um Dateigrößen vor dem Senden an den
    Client zu verringern.
  \end{itemize}
\item
  \textbf{Verbessere die Datenbankleistung:}

  \begin{itemize}
  \tightlist
  \item
    Optimiere Abfragen, setze Indizes ein und überlege, einen Cache wie
    Redis oder Memcached zu verwenden.
  \end{itemize}
\item
  \textbf{Mobile Optimierung:}

  \begin{itemize}
  \tightlist
  \item
    Stelle sicher, dass deine Website für mobile Geräte optimiert ist,
    da immer mehr Nutzer über Mobilgeräte zugreifen.
  \end{itemize}
\item
  \textbf{Überwache und messe regelmäßig:}

  \begin{itemize}
  \tightlist
  \item
    Verwende Tools wie Google PageSpeed Insights, GTmetrix oder
    WebPageTest, um die Performance deiner Website regelmäßig zu
    überprüfen.
  \end{itemize}
\end{enumerate}

\hypertarget{links}{%
\subsection{Links}\label{links}}

\begin{itemize}
\item
  SVG \url{https://svgoptimizer.com/de/}
\item
  MOV to MP4/WEBM Converter \url{https://cloudconvert.com/mov-to-webm}
\item
  XAMPP Apache + MariaDB + PHP
  \url{https://www.apachefriends.org/de/index.html}
\end{itemize}

\hypertarget{webdesign}{%
\subsection{Webdesign}\label{webdesign}}

\begin{itemize}
\item
  Download-Link für eine PDF-Datei
\item
  Footer und die Bilder mittig positionieren und den Bildnamen
  anzuzeigen
\item
  Darstellung der HTML {\lstinline!<pre><code>!} Umgebung
  anpassen, Quellcode linksbündig darstellen und breite Texte mit
  Zeilenumbrüchen und Einrückungen versehen
\item
  Hover- und Fokus-Stil für Links und Bilder, Bilder nebeneinander
  anordnen, Bildgöße
\item
  CSS-Anweisungen optimieren und CSS-Variablen verwenden, um den Code
  sauberer und wartbarer zu machen
\item
  Roboto-Font aus Google Fonts
\item
  Inhalt sowohl horizontal als auch vertikal zu zentrieren und sich
  dabei am oberen Rand des Ansichtsfensters zu befinden
\end{itemize}

Browser-Caching ist ein effektives Mittel, um die Ladegeschwindigkeit
einer Website zu verbessern.

\begin{lstlisting}[language=bash]
# .htaccess-Datei für Apache-Server:
<IfModule mod_expires.c>
ExpiresActive On
ExpiresByType image/jpg "access 1 year"
ExpiresByType image/jpeg "access 1 year"
ExpiresByType image/gif "access 1 year"
ExpiresByType image/png "access 1 year"
ExpiresByType text/css "access 1 month"
ExpiresByType text/html "access 1 month"
ExpiresByType application/pdf "access 1 month"
ExpiresByType text/x-javascript "access 1 month"
ExpiresByType application/x-shockwave-flash "access 1 month"
ExpiresByType image/x-icon "access 1 year"
ExpiresDefault "access 1 month"
</IfModule>
\end{lstlisting}

E-Tag (Entity Tag) ist ein Teil des HTTP-Protokolls, das von Webservern
und Browsern verwendet wird, um die Gültigkeit einer bestimmten Version
einer Ressource zu überprüfen.

\begin{lstlisting}[language=bash]
# .htaccess-Datei für Apache-Server:
FileETag MTime Size
\end{lstlisting}

\hypertarget{webdesign-merkmale-von-spiegel}{%
\subsection{Webdesign-Merkmale von
Spiegel}\label{webdesign-merkmale-von-spiegel}}

Die Website von ``spiegel.de'' ist die Online-Ausgabe von Deutschlands
bekanntem Nachrichtenmagazin, dem SPIEGEL. Wie bei vielen
Nachrichtenportalen, weist sie bestimmte Design- und Funktionsmerkmale
auf, um den Inhalt klar und ansprechend zu präsentieren. Hier sind
einige charakteristische Webdesign-Merkmale, die man auf solchen großen
Nachrichtenportalen häufig findet:

\begin{enumerate}
\def\labelenumi{\arabic{enumi}.}
\item
  \textbf{Responsives Design}: Die Website ist so gestaltet, dass sie
  auf verschiedenen Geräten -- von Mobiltelefonen über Tablets bis hin
  zu Desktops -- gut aussieht und funktioniert.
\item
  \textbf{Header mit Branding}: Oft gibt es ein prominent platziertes
  Logo und Navigationsmenü am oberen Rand der Seite.
\item
  \textbf{Mehrspaltiges Layout}: Auf der Startseite und in den
  Kategorieseiten finden sich oft mehrere Spalten mit verschiedenen
  Nachrichtenkategorien oder Schlagzeilen.
\item
  \textbf{Bilder und Multimedia}: Große, auffällige Bilder und
  eingebettete Videos sind häufig zu finden, um Geschichten zu ergänzen
  und die Aufmerksamkeit der Leser zu fesseln.
\item
  \textbf{Inhaltskarten}: Artikel können in ``Karten'' dargestellt
  werden, die einen Titel, ein Bild und vielleicht eine kurze
  Beschreibung enthalten.
\item
  \textbf{Scrollbare Inhalte}: Besonders auf der Startseite gibt es oft
  unendliches Scrollen oder ``Lazy Loading'', bei dem neue Inhalte
  geladen werden, wenn der Benutzer zum Ende der Seite scrollt.
\item
  \textbf{Footer-Bereich}: Hier finden sich oft Links zu Impressum,
  Datenschutzerklärung, Kontaktinformationen und anderen wichtigen
  Seiten.
\item
  \textbf{Interaktive Elemente}: Dazu gehören Dropdown-Menüs,
  Suchfelder, Buttons für soziale Medien und mehr.
\item
  \textbf{Werbung}: Große Nachrichtenseiten haben oft Display-Werbung in
  verschiedenen Formaten und Größen.
\item
  \textbf{Dunkler Modus}: Einige moderne Websites, einschließlich
  Nachrichtenportalen, bieten einen ``dunklen Modus'' an, der besonders
  in Umgebungen mit wenig Licht angenehm für die Augen ist.
\item
  \textbf{Eingebettete Widgets}: Dazu können Live-Ticker, Umfragen oder
  interaktive Grafiken gehören.
\end{enumerate}

Es ist wichtig zu beachten, dass die spezifischen Designentscheidungen,
die für ``spiegel.de'' getroffen wurden, auf einer Vielzahl von Faktoren
basieren, einschließlich Markenidentität, Benutzererfahrung und
Geschäftszielen. Wenn Sie das Design einer bestimmten Website nachahmen
oder sich davon inspirieren lassen möchten, ist es ratsam, diese
Faktoren zu berücksichtigen und sicherzustellen, dass Ihr Design den
Bedürfnissen und Erwartungen Ihrer Zielgruppe entspricht.

\hypertarget{schreibstile-1}{%
\subsection{Schreibstile}\label{schreibstile-1}}

\begin{lstlisting}[language=bash]
# Favorit: Expositorisch
# Schreibstile: Beschreibend, Expositorisch, Reflektierend, Akademisch, Kritisch
Schreibstil: Expositorisch
Erstellen Sie je nach Schreibstil eine kurze (ca. 35 Wörter) und ansprechende Zusammenfassung des folgenden Artikels. Die Zusammenfassung sollte für jemanden ohne wissenschaftlichen Hintergrund verständlich sein und gleichzeitig die wichtigsten Informationen genau wiedergeben. 
Artikel: " "
\end{lstlisting}

Es gibt zahlreiche Schreibstile, die je nach Kontext, Zweck und Publikum
variieren können. Hier sind einige der gängigsten Schreibstile:

\begin{enumerate}
\def\labelenumi{\arabic{enumi}.}
\item
  \textbf{Erzählerisch (narrativ):} Dieser Stil erzählt eine Geschichte,
  sei es Fiktion oder Non-Fiktion.
\item
  \textbf{Beschreibend:} Hierbei wird der Fokus auf detaillierte
  Beschreibungen gelegt, um ein Bild oder ein Gefühl für den Leser zu
  erzeugen.
\item
  \textbf{Überzeugend (argumentativ):} Der Zweck ist es, den Leser von
  einer bestimmten Meinung, Idee oder Perspektive zu überzeugen.
\item
  \textbf{Expositorisch:} Dieser Stil informiert oder erläutert ein
  Thema. Er wird oft in akademischen und journalistischen Kontexten
  verwendet.
\item
  \textbf{Reflektierend:} Hier reflektiert der Autor über persönliche
  Erfahrungen und Erlebnisse, oft in einem Tagebuch oder in persönlichen
  Essays.
\item
  \textbf{Direktiv (imperativ):} Dieser Stil gibt Anweisungen oder
  Anordnungen, wie in Kochrezepten oder Bedienungsanleitungen.
\item
  \textbf{Lyrisch:} Hierbei handelt es sich um einen poetischen
  Schreibstil, der oft in Gedichten, Liedtexten und manchmal in Prosa
  vorkommt.
\item
  \textbf{Kritisch:} Analyse und Bewertung eines Themas oder Werkes, oft
  in Buchkritiken oder Filmrezensionen.
\item
  \textbf{Journalistisch:} Hierbei geht es um das objektive Berichten
  von Nachrichten und aktuellen Ereignissen.
\item
  \textbf{Akademisch:} Ein formeller Stil, der in wissenschaftlichen
  Arbeiten, Aufsätzen und Forschungsartikeln verwendet wird.
\item
  \textbf{Dialogisch:} Dieser Stil betont den Dialog zwischen
  Charakteren oder zwischen Autor und Leser.
\item
  \textbf{Satirisch:} Ein humorvoller oder ironischer Stil, der oft dazu
  verwendet wird, Kritik zu üben oder auf Probleme aufmerksam zu machen.
\item
  \textbf{Epistolisch:} Ein Stil, der in Briefen oder Korrespondenzen
  verwendet wird.
\end{enumerate}

Es ist wichtig zu beachten, dass viele Texte oft eine Kombination aus
verschiedenen Stilen enthalten können, je nach Bedarf und Kontext des
Schreibens.

\hypertarget{bewertung-von-html-datei-und-css-code}{%
\section{Bewertung von HTML-Datei und
CSS-Code}\label{bewertung-von-html-datei-und-css-code}}

\textbf{HTML-Datei}

Unter Berücksichtigung Ihrer Anforderungen und basierend auf der
bereitgestellten HTML-Datei sind hier einige Anmerkungen und
Verbesserungen, um eine effektive, barrierefreie und responsive Webseite
zu erstellen:

\begin{enumerate}
\def\labelenumi{\arabic{enumi}.}
\tightlist
\item
  \textbf{Best Practices:}

  \begin{itemize}
  \tightlist
  \item
    Sie haben bereits das HTML5 DOCTYPE, die charset Deklaration, und
    Meta-Viewport-Tags, das ist großartig.
  \end{itemize}
\item
  \textbf{Barrierefreiheit:}

  \begin{itemize}
  \tightlist
  \item
    Die Verwendung von {\lstinline!role!} Attributen ist
    gut, diese können in einigen Fällen jedoch redundant sein, da
    HTML5-Elemente wie {\lstinline!<header>!},
    {\lstinline!<nav>!}, {\lstinline!<main>!}
    und {\lstinline!<footer>!} bereits diese semantischen
    Rollen implizieren.
  \item
    Stellen Sie sicher, dass der Kontrast zwischen Text- und
    Hintergrundfarben ausreichend ist.
  \item
    Verwenden Sie ARIA-Attribute sparsam und nur dann, wenn sie
    notwendig sind.
  \item
    Es wäre auch gut, wenn Sie andere semantische HTML-Elemente wie
    {\lstinline!<article>!},
    {\lstinline!<section>!} oder
    {\lstinline!<aside>!} verwenden würden, wo es sinnvoll
    ist.
  \end{itemize}
\item
  \textbf{Responsive Design:}

  \begin{itemize}
  \tightlist
  \item
    Sie haben den Meta-Viewport-Tag verwendet, das ist sehr wichtig für
    ein Responsive Design. Der CSS-Code, den Sie zuvor bereitgestellt
    haben, enthält bereits Media-Queries für verschiedene
    Bildschirmgrößen.
  \end{itemize}
\item
  \textbf{Text vertikal zentrieren:}

  \begin{itemize}
  \item
    Sie können Flexbox oder Grid verwenden, um den Inhalt innerhalb von
    {\lstinline!.container!} vertikal zu zentrieren. Hier
    ist eine Möglichkeit mit Flexbox:
  \item
\begin{lstlisting}
.container {
   display: flex;
   flex-direction: column;
   justify-content: center;
   height: 100vh; /* Mindestens die Höhe des Viewports */
}
\end{lstlisting}
  \end{itemize}
\item
  \textbf{Maßeinheiten in em:}

  \begin{itemize}
  \tightlist
  \item
    Die Verwendung von {\lstinline!em!} für Ihre
    Maßeinheiten ist bereits in Ihrem CSS umgesetzt.
  \end{itemize}
\item
  \textbf{Native Lazy Loading:}

  \begin{itemize}
  \tightlist
  \item
    Wenn Sie Bilder hinzufügen, können Sie das
    {\lstinline!loading="lazy"!} Attribut verwenden, um
    native Lazy Loading zu aktivieren. Zum Beispiel:
  \item
    {\lstinline!<img src="path-to-image.jpg" alt="Beschreibung des Bildes" loading="lazy">!}
  \end{itemize}
\item
  \textbf{Performance:}

  \begin{itemize}
  \tightlist
  \item
    Verkleinern und komprimieren Sie Ihre CSS- und JS-Dateien.
  \item
    Überlegen Sie, ob Sie ein Tool wie
    \href{https://developers.google.com/web/tools/lighthouse}{Lighthouse}
    verwenden möchten, um Ihre Seite auf Performance-Probleme zu
    überprüfen.
  \item
    Optimieren Sie Bilder für das Web.
  \item
    Sie könnten auch in Erwägung ziehen, eine Content Delivery Network
    (CDN) zu verwenden, um Ihre Webseite schneller zu laden.
  \end{itemize}
\end{enumerate}

Abschließend sieht Ihr HTML-Grundgerüst schon recht solide aus. Sie
sollten jedoch weiterhin Tools wie den
\href{https://wave.webaim.org/}{WAVE Accessibility Tool} oder
\href{https://developers.google.com/web/tools/lighthouse}{Lighthouse}
verwenden, um sicherzustellen, dass Ihre Webseite sowohl barrierefrei
als auch leistungsstark ist.

\textbf{CSS-Code}

Ihr CSS-Code scheint gut strukturiert und durchdacht zu sein. Es sind
viele Best Practices implementiert. Hier sind jedoch einige Anmerkungen
und Optimierungen, die Sie berücksichtigen könnten:

\begin{enumerate}
\def\labelenumi{\arabic{enumi}.}
\tightlist
\item
  \textbf{Best Practices:}

  \begin{itemize}
  \tightlist
  \item
    Sie verwenden CSS-Variablen, was ein ausgezeichneter Ansatz ist, um
    den Code wartbarer zu machen.
  \item
    Sie haben die Schatten, Farben und Schriftarten zentralisiert, was
    eine konsistente Gestaltung erleichtert.
  \end{itemize}
\item
  \textbf{Barrierefreiheit:}

  \begin{itemize}
  \tightlist
  \item
    Sie haben einen Fokus-Stil für Links hinzugefügt, was gut für die
    Barrierefreiheit ist. Es könnte hilfreich sein, auch Fokus-Stile für
    andere interaktive Elemente hinzuzufügen.
  \end{itemize}
\item
  \textbf{Responsive Design:}

  \begin{itemize}
  \tightlist
  \item
    Ihre Medienabfragen sehen solide aus, und Sie haben verschiedene
    Stile für verschiedene Bildschirmgrößen definiert.
  \end{itemize}
\item
  \textbf{Vertikale Zentrierung:}

  \begin{itemize}
  \tightlist
  \item
    Sie haben das {\lstinline!display: flex!} verwendet,
    aber Sie haben nicht wirklich das vertikale Zentrieren von Text in
    Containern implementiert. Sie können dies durch die Verwendung von
    {\lstinline!align-items: center!} in flex-Containern
    erreichen.
  \end{itemize}
\item
  \textbf{Maßeinheiten in em:}

  \begin{itemize}
  \tightlist
  \item
    Sie haben {\lstinline!em!}-Einheiten konsequent
    verwendet, was gut ist. Beachten Sie jedoch, dass die
    Basis-Schriftgröße ({\lstinline!--base-font-size!}) in
    {\lstinline!px!} definiert ist. Wenn Sie die volle
    Flexibilität von {\lstinline!em!} nutzen möchten,
    könnten Sie die Basis-Schriftgröße in {\lstinline!rem!}
    oder {\lstinline!\%!} definieren.
  \end{itemize}
\item
  \textbf{Native Lazy Loading:}

  \begin{itemize}
  \tightlist
  \item
    Sie haben keine direkte Implementierung von Lazy Loading für Bilder.
    Um native Lazy Loading zu implementieren, fügen Sie
    {\lstinline!loading="lazy"!} zum
    {\lstinline!img!}-Tag hinzu:
  \item
    {\lstinline!<img src="path/to/image.jpg" alt="Beschreibung" loading="lazy">!}
  \end{itemize}
\item
  \textbf{Performance:}

  \begin{itemize}
  \tightlist
  \item
    Sie haben Schatten und Übergänge verwendet, die die Performance
    beeinflussen könnten, besonders auf mobilen Geräten. Es ist jedoch
    schwer zu sagen, ohne die gesamte Seite in Aktion zu sehen. Tools
    wie
    \href{https://developers.google.com/speed/pagespeed/insights/}{Google
    PageSpeed Insights} können hilfreich sein, um dies zu überprüfen.
  \item
    Achten Sie darauf, externe Schriftarten (wie Roboto) effizient zu
    laden, um die Webseiten-Performance nicht zu beeinträchtigen.
  \end{itemize}
\item
  \textbf{Sonstige Anmerkungen:}

  \begin{itemize}
  \tightlist
  \item
    Die {\lstinline!.responsive-video!} Klasse ist
    definiert, aber es fehlen Stile, die sie wirklich responsiv machen.
    Sie könnten einen Wrapper mit einem festen Verhältnis verwenden und
    das Video darin absolut positionieren, um es responsiv zu machen.
  \item
    Die {\lstinline!--responsive-img-max-width!} Variable
    wird im CSS nicht verwendet. Wenn Sie diese Variable für die
    maximale Breite von Bildern verwenden möchten, stellen Sie sicher,
    dass sie in Ihrem Code implementiert ist.
  \item
    Achten Sie darauf, dass Sie Alt-Texte für alle Bilder verwenden, um
    die Barrierefreiheit und SEO zu verbessern.
  \end{itemize}
\end{enumerate}

Insgesamt haben Sie bereits einen guten Ansatz gewählt, aber wie bei
allem im Webdesign gibt es immer Raum für Verbesserungen und
Anpassungen, abhängig von den spezifischen Anforderungen und Zielen
Ihrer Webseite. % Platzhalter

\end{document}
