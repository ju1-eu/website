% vorlage-main.tex
\documentclass[12pt,a4paper]{scrartcl}
\usepackage{vorlage-design-main}% vorlage-design-main.sty
% Bildgröße global ändern
\setkeys{Gin}{width=0.75\linewidth}
% Literatur
\addbibresource{referenzen.bib}
% Anpassung des Quellcode-Stils
\lstset{
  basicstyle=\ttfamily\small,   % Ändern Sie die Schriftgröße des Quellcodes, falls erforderlich.
  columns=fullflexible,
  breaklines=true,              % Zeilenumbrüche für zu lange Zeilen
  postbreak=\mbox{$\hookrightarrow$\space}, % Pfeil am Zeilenumbruch
  literate={ö}{{\"o}}1
           {ä}{{\"a}}1
           {ü}{{\"u}}1
           {Ö}{{\"O}}1
           {Ä}{{\"A}}1
           {Ü}{{\"U}}1
           {ß}{{\ss}}1,
  xleftmargin=2em,              % Optional: Linker Abstand
  xrightmargin=3em,             % Optional: Rechter Abstand
  showstringspaces=false,
  showspaces=false             % Zeigt keine Leerzeichen an
}
% Hyperlinks
\hypersetup{
    colorlinks=true,
    linkcolor=blue,
    filecolor=magenta,
    urlcolor=cyan,
}
% Fehler wenn pandoc - Markdown in Latex
\newcommand{\tightlist}{
  \setlength{\itemsep}{0pt}\setlength{\parskip}{0pt}
}
% Titel, Autor und Datum
\title{Mein optimiertes Dokument}
%\author{Jan Unger}
\date{\today}

\begin{document}
\maketitle

\hypertarget{ohmsche-gesetz}{%
\section{Ohm`sche Gesetz}\label{ohmsche-gesetz}}

\begin{lstlisting}[language=Python]
# Quellcode in Python
import math
import matplotlib 
matplotlib.rcParams['text.usetex'] = True # Latex code
import matplotlib.pyplot as plt 

R1 = 1 # Ohm
R2 = 2 # Ohm
R3 = 3 # Ohm
N=500 
X=[x / 10 for x in range(N)] # Ampere
Y1=[R1 * i for i in X]  # Spannung
Y2=[R2 * i for i in X]  # Spannung
Y3=[R3 * i for i in X]  # Spannung


# Latex
# Farben: Orange #F28C64 und grau2 #B2B2B2
plt.plot(X,Y1, label=r'$R_1 = 1~\Omega$', color="black")
plt.plot(X,Y2, label=r'$R_2 = 2~\Omega$', color="#A71916")#rot5 #A71916
plt.plot(X,Y3, label=r'$R_3 = 3~\Omega$', color="#0D468E")#blau5 #0D468E
plt.title(r'Ohmsche Gesetz $U = R \times I$',fontsize=12)
plt.xlabel(r'\textbf{Strom (A)}')
plt.ylabel(r'\textbf{Spannung (V)}')
plt.xlim(0,+10) 
plt.ylim(0,+10)
plt.legend()
plt.savefig("Diag_Ohmsche_Gesetz.svg")
plt.show()
\end{lstlisting}

\includesvg{../images/Diag_Ohmsche_Gesetz-min.svg}  % Platzhalter

\end{document}
