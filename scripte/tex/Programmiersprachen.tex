% vorlage-main.tex
\documentclass[12pt,a4paper]{scrartcl}
\usepackage{vorlage-design-main}% vorlage-design-main.sty
% Bildgröße global ändern
\setkeys{Gin}{width=0.75\linewidth}
% Literatur
\addbibresource{referenzen.bib}
% Anpassung des Quellcode-Stils
\lstset{
  basicstyle=\ttfamily\small,   % Ändern Sie die Schriftgröße des Quellcodes, falls erforderlich.
  columns=fullflexible,
  breaklines=true,              % Zeilenumbrüche für zu lange Zeilen
  postbreak=\mbox{$\hookrightarrow$\space}, % Pfeil am Zeilenumbruch
  literate={ö}{{\"o}}1
           {ä}{{\"a}}1
           {ü}{{\"u}}1
           {Ö}{{\"O}}1
           {Ä}{{\"A}}1
           {Ü}{{\"U}}1
           {ß}{{\ss}}1,
  xleftmargin=2em,              % Optional: Linker Abstand
  xrightmargin=3em,             % Optional: Rechter Abstand
  showstringspaces=false,
  showspaces=false             % Zeigt keine Leerzeichen an
}
% Hyperlinks
\hypersetup{
    colorlinks=true,
    linkcolor=blue,
    filecolor=magenta,
    urlcolor=cyan,
}
% Fehler wenn pandoc - Markdown in Latex
\newcommand{\tightlist}{
  \setlength{\itemsep}{0pt}\setlength{\parskip}{0pt}
}
% Titel, Autor und Datum
\title{Mein optimiertes Dokument}
%\author{Jan Unger}
\date{\today}

\begin{document}
\maketitle

\hypertarget{programmiersprachen}{%
\section{Programmiersprachen}\label{programmiersprachen}}

Aufgaben zum C/C++ Kurs / Universität Regensburg / Fakultät Physik

\begin{lstlisting}
# ChatGPT
Aufgabe: " "

Entwickeln Sie ein Programm in den Programmiersprachen C, C++, Python, Arduino und PHP. 

Das Programm soll: Bei fehlerhaften Eingaben (z.B. wenn der Benutzer keine Zahl eingibt) 
sollte das Programm den Benutzer darauf hinweisen und ihn erneut zur Eingabe auffordern.

Anforderungen:

-   Implementieren Sie Fehlerprüfungen für die Eingabe.
-   Kommentieren Sie den Code auf Deutsch, wo es sinnvoll und notwendig ist.
-   Befolgen Sie die Best Practices

Nachdem Sie das Programm in allen fünf Programmiersprachen erstellt haben, 
dokumentieren Sie die Testergebnisse in Markdown.
\end{lstlisting}

\hypertarget{code-auszufuxfchren}{%
\section{Code auszuführen}\label{code-auszufuxfchren}}

\hypertarget{c}{%
\subsubsection{C:}\label{c}}

\begin{enumerate}
\def\labelenumi{\arabic{enumi}.}
\tightlist
\item
  Installieren Sie einen C-Compiler, z. B. GCC.
\item
  Speichern Sie den Code in einer Datei namens
  {\lstinline!hello.c!}.
\item
  Öffnen Sie das Terminal oder die Kommandozeile.
\item
  Kompilieren Sie den Code mit:
  {\lstinline!gcc hello.c -o hello!}
\item
  Führen Sie das Programm aus: {\lstinline!./hello!} (oder
  {\lstinline!hello.exe!} auf Windows).
\end{enumerate}

\hypertarget{c-1}{%
\subsubsection{C++:}\label{c-1}}

\begin{enumerate}
\def\labelenumi{\arabic{enumi}.}
\tightlist
\item
  Installieren Sie einen C++-Compiler, z. B. G++.
\item
  Speichern Sie den Code in einer Datei namens
  {\lstinline!hello.cpp!}.
\item
  Öffnen Sie das Terminal oder die Kommandozeile.
\item
  Kompilieren Sie den Code mit:
  {\lstinline!g++ hello.cpp -o hello!}
\item
  Führen Sie das Programm aus: {\lstinline!./hello!} (oder
  {\lstinline!hello.exe!} auf Windows).
\end{enumerate}

\hypertarget{python}{%
\subsubsection{Python:}\label{python}}

\begin{enumerate}
\def\labelenumi{\arabic{enumi}.}
\tightlist
\item
  Installieren Sie Python von
  \href{https://www.python.org/}{python.org}.
\item
  Speichern Sie den Code in einer Datei namens
  {\lstinline!hello.py!}.
\item
  Öffnen Sie das Terminal oder die Kommandozeile.
\item
  Führen Sie das Skript aus: {\lstinline!python hello.py!}
\end{enumerate}

\hypertarget{arduino}{%
\subsubsection{Arduino:}\label{arduino}}

\begin{enumerate}
\def\labelenumi{\arabic{enumi}.}
\tightlist
\item
  Installieren Sie die Arduino IDE von der
  \href{https://www.arduino.cc/}{Arduino-Website}.
\item
  Öffnen Sie die Arduino IDE und kopieren Sie den Code in das Textfeld.
\item
  Verbinden Sie Ihren Arduino über USB.
\item
  Wählen Sie unter {\lstinline!Werkzeuge!} -\textgreater{}
  {\lstinline!Port!} den entsprechenden COM-Port aus.
\item
  Klicken Sie auf {\lstinline!Hochladen!}.
\item
  Öffnen Sie den Seriellen Monitor, um Eingaben zu tätigen und Ausgaben
  zu sehen.
\end{enumerate}

\hypertarget{php}{%
\subsubsection{PHP:}\label{php}}

\begin{enumerate}
\def\labelenumi{\arabic{enumi}.}
\tightlist
\item
  Installieren Sie PHP von \href{https://www.php.net/}{php.net}.
\item
  Speichern Sie den Code in einer Datei namens
  {\lstinline!hello.php!}.
\item
  Öffnen Sie das Terminal oder die Kommandozeile.
\item
  Führen Sie das Skript aus: {\lstinline!php hello.php!}
\end{enumerate}

\hypertarget{makefile}{%
\section{Makefile}\label{makefile}}

Ein {\lstinline!Makefile!} ist ein nützliches Tool, um
Build-Prozesse zu automatisieren. Es ist besonders hilfreich, wenn Sie
mehrere Dateien haben oder Abhängigkeiten zwischen diesen Dateien
bestehen. Für den bereitgestellten Code (eine einfache
{\lstinline!hello.c!} für C und
{\lstinline!hello.cpp!} für C++) kann ein
{\lstinline!Makefile!} recht einfach sein.

Hier ist ein {\lstinline!Makefile!}, das sowohl für den C-
als auch für den C++-Code verwendet werden kann:

\begin{lstlisting}[language=make]
# Compiler und Flags definieren
CC = gcc
CXX = g++
CFLAGS = -std=c17 -Wall -O2
CXXFLAGS = -std=c++20 -Wall -O2
LDFLAGS = 

# Ziel- und Quell-Dateien definieren
C_SOURCES = main.c
CPP_SOURCES = main.cpp
C_OBJECTS = $(C_SOURCES:.c=.o)
CPP_OBJECTS = $(CPP_SOURCES:.cpp=.o)
TARGET_C = main_c
TARGET_CPP = main_cpp

# Regel, um das C-Programm zu erstellen
$(TARGET_C): $(C_OBJECTS)
    $(CC) $(C_OBJECTS) -o $(TARGET_C) $(LDFLAGS)

# Regel, um das C++-Programm zu erstellen
$(TARGET_CPP): $(CPP_OBJECTS)
    $(CXX) $(CPP_OBJECTS) -o $(TARGET_CPP) $(LDFLAGS)

# Regel, um C-Objekt-Dateien zu erstellen
%.o: %.c
    $(CC) -c $(CFLAGS) $< -o $@

# Regel, um C++-Objekt-Dateien zu erstellen
%.o: %.cpp
    $(CXX) -c $(CXXFLAGS) $< -o $@

# Zusätzliche nützliche Regeln
clean:
    rm -f $(C_OBJECTS) $(CPP_OBJECTS) $(TARGET_C) $(TARGET_CPP)

all: $(TARGET_C) $(TARGET_CPP)
\end{lstlisting}

Um die Programme zu kompilieren, können Sie folgende Befehle verwenden:

\begin{itemize}
\tightlist
\item
  Für das C-Programm: {\lstinline!make $(TARGET\_C)!}
\item
  Für das C++-Programm: {\lstinline!make $(TARGET\_CPP)!}
\item
  Für beide Programme: {\lstinline!make all!}
\item
  Zum Bereinigen der kompilierten Dateien:
  {\lstinline!make clean!}
\end{itemize}

Stellen Sie sicher, dass Sie die Variablen
{\lstinline!C\_SOURCES, CPP\_SOURCES, TARGET\_C und TARGET\_CPP!}
an Ihren eigenen Code anpassen. % Platzhalter

\end{document}
