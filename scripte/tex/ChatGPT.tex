% vorlage-main.tex
\documentclass[12pt,a4paper]{scrartcl}
\usepackage{vorlage-design-main}% vorlage-design-main.sty
% Bildgröße global ändern
\setkeys{Gin}{width=0.75\linewidth}
% Literatur
\addbibresource{referenzen.bib}
% Anpassung des Quellcode-Stils
\lstset{
  basicstyle=\ttfamily\small,   % Ändern Sie die Schriftgröße des Quellcodes, falls erforderlich.
  columns=fullflexible,
  breaklines=true,              % Zeilenumbrüche für zu lange Zeilen
  postbreak=\mbox{$\hookrightarrow$\space}, % Pfeil am Zeilenumbruch
  literate={ö}{{\"o}}1
           {ä}{{\"a}}1
           {ü}{{\"u}}1
           {Ö}{{\"O}}1
           {Ä}{{\"A}}1
           {Ü}{{\"U}}1
           {ß}{{\ss}}1,
  xleftmargin=2em,              % Optional: Linker Abstand
  xrightmargin=3em,             % Optional: Rechter Abstand
  showstringspaces=false,
  showspaces=false             % Zeigt keine Leerzeichen an
}
% Hyperlinks
\hypersetup{
    colorlinks=true,
    linkcolor=blue,
    filecolor=magenta,
    urlcolor=cyan,
}
% Fehler wenn pandoc - Markdown in Latex
\newcommand{\tightlist}{
  \setlength{\itemsep}{0pt}\setlength{\parskip}{0pt}
}
% Titel, Autor und Datum
\title{Mein optimiertes Dokument}
%\author{Jan Unger}
\date{\today}

\begin{document}
\maketitle

\hypertarget{chatgpt-3.54}{%
\section{ChatGPT-3.5/4}\label{chatgpt-3.54}}

ChatGPT \url{https://chat.openai.com/}

update: 2-8-23

\textbf{Allgemein}

\begin{itemize}
\tightlist
\item
  {\lstinline!Übersetze das auf Englisch!} oder
  \url{https://www.deepl.com/write}
\item
  {\lstinline!Fasse diesen Paragrafen zusammen!} oder
  \textbf{Merlin}
\item
  {\lstinline!Bitte schreibe einen Aufsatz mit zwei Absätzen Länge zu diesem Thema!}
\item
  {\lstinline!Schreibe mir im Python-Code Hallo Welt!}
\item
  {\lstinline!Erstelle mir einen tabellarischen Vergleich mit Merkmalen und Vorteilen!}
\item
  {\lstinline!Rechne ein Beispiel!}
\item
  {\lstinline!Erkläre mir das Thema anhand des Buches!}
\item
  {\lstinline!Erkläre mir das Thema!}
\item
  {\lstinline!Fasse mir die drei wichtigsten Punkte zusammen zu deiner Erklärung!}
\item
  {\lstinline!Gib mir eine kommagetrennte Liste mit den wichtigsten keywords zu deiner Erklärung!}
\item
  {\lstinline!Erstelle mir bitte 5 Prüfungsfragen mit Lösungen zu deiner Erklärung!}
\end{itemize}

\hypertarget{mathematik-elektrik-schaltplan-python}{%
\section{Mathematik + Elektrik + Schaltplan +
Python}\label{mathematik-elektrik-schaltplan-python}}

\begin{lstlisting}
# ChatGPT
**Aufgabenstellung: Elektrische Eigenschaften einer Reihenschaltung von Widerständen**

**Teil A:** 
Entwickle ein Python-Programm, das die elektrischen Eigenschaften einer Reihenschaltung von drei Widerständen berechnet.

**Eingabe:** 
Drei Widerstandswerte in Ohm und eine Spannungsquelle von 12 V.

**Gesuchte Ausgaben:**

- Teilspannungen über jeden Widerstand.
- Gesamtstrom durch die Schaltung.
- Gesamtleistung der Schaltung.
- Leistung über jeden Widerstand.
- Gesamtwiderstand der Schaltung.

Nutze das Modul `matplotlib` für Python, um die Ergebnisse grafisch darzustellen. Teste dein Programm anschließend mit drei exemplarischen Widerstandswerten.

**Teil B:** 
Erläutere die Berechnungen für eine Reihenschaltung von Widerständen mit Hilfe von Markdown.

**Inhalt:**

- Beschreibung der gegebenen Werte und der gesuchten Größen.
- Manuelle Berechnung der oben genannten Größen.
- Darstellung der Reihenschaltung mit Hilfe eines einfachen ASCII-Schaltplans.

Verwende die gegebenen Formeln und Lösungen, um die Teile A und B entsprechend auszuarbeiten.
\end{lstlisting}

\hypertarget{website}{%
\section{Website}\label{website}}

\textbf{Erstelle und Strukturiere mir Webseiteninhalte}

\begin{lstlisting}
# ChatGPT
**Aufgabenstellung 1: Erstelle und Strukturiere mir Webseiteninhalte**

Ziel dieser Aufgabe ist die Erstellung und Strukturierung von Dateien für eine Website gemäß den angegebenen Spezifikationen.

**Anforderungen:**

1. **Root-Ordner (Hauptverzeichnis):** 
   - Erstellen Sie eine `index.html` im Root-Ordner der Webseite.

2. **CSS-Ordner:** 
   - Erstellen Sie eine CSS-Datei im Ordner `css`. Die genaue Bezeichnung der Datei kann frei gewählt werden, sollte jedoch beschreibend sein, z.B. `main.css`.
   
3. **Bilder-Ordner:** 
   - Platzieren Sie einen Bildplatzhalter im Ordner `img`. Dies kann ein Standardbild sein, das später durch ein tatsächliches Bild ersetzt wird.

4. **PHP-Ordner:** 
   - Erstellen Sie eine PHP-Datei im Ordner `php`. Der Name der Datei sollte den Inhalt oder den Zweck der Datei widerspiegeln.

5. **HTML-Ordner:** 
   - Erstellen Sie eine zusätzliche HTML-Datei im Ordner `html`. Die Bezeichnung dieser Datei kann frei gewählt werden, sollte jedoch den Inhalt oder Zweck der Datei deutlich machen.

**Hinweise:**
   - Achten Sie auf eine klare und saubere Strukturierung der Dateien und Ordner.
   - Beachten Sie die Best Practices bei der Benennung von Dateien und Ordnern, um Verwirrung oder Konflikte zu vermeiden.

**Abgabe:** 
   - Ein Zip-Archiv oder einen Link zu einem Repository, das alle erstellten Dateien und Ordner enthält.
   - Ein kurzer Bericht (optional), in dem die Struktur und der Zweck jeder Datei beschrieben wird.

**Deadline:** 
   - Abgabedatum: 1.9.2023

**Bewertungskriterien:** 
   - die Qualität des Codes, die Einhaltung der Best Practices und die Ästhetik des Designs.
\end{lstlisting}

\textbf{Überprüfe und Optimiere die Webseiteninhalte}

\begin{lstlisting}
# ChatGPT
**Aufgabenstellung 2: Überprüfe und Optimiere die Webseiteninhalte**

Ziel dieser Aufgabe ist die Analyse und Optimierung von Webinhalten, um eine bestmögliche Benutzererfahrung über verschiedene Geräte hinweg sicherzustellen. 

**Anforderungen:**

1. **Best Practices für html, CSS und php:** 
   - Untersuchen Sie das aktuelle CSS und vergewissern Sie sich, dass es den gängigen Best Practices entspricht.
   
2. **Barrierefreiheit:** 
   - Stellen Sie sicher, dass die Webseite für alle Benutzer zugänglich ist, einschließlich Menschen mit Behinderungen.
   
3. **Responsive Design:** 
   - Die Webseite sollte auf unterschiedlichen Geräten - von Mobiltelefonen über Tablets bis hin zu Desktop-Computern - gut lesbar und nutzbar sein.
   
4. **Einheitlichkeit der Maßeinheiten:** 
   - Verwenden Sie, wo immer möglich, die Maßeinheit "em" in Ihrem CSS, um Skalierbarkeit und Flexibilität über verschiedene Geräte und Bildschirmgrößen hinweg sicherzustellen.
   
5. **Ziel-Dateien:** 
   - Die Überprüfung und Anpassungen sollten sich auf "Index.html" sowie die zugehörigen PHP-Seiten erstrecken.

**Abgabe:** 
   - Überarbeitetes CSS-Dokument
   - Überarbeitete index.html und PHP-Dateien, falls notwendig
   - Ein kurzer Bericht über die vorgenommenen Änderungen und deren Auswirkungen auf die Benutzererfahrung.
\end{lstlisting}

\textbf{Kontrolliere nochmals}

\begin{lstlisting}
# ChatGPT
Nehmen Sie die Rolle eines erfahrenen Webprogrammierer und Webdesigner ein.

Aufgabe: Kontrolliere nochmals unter Berücksichtigung von

- Best Practices  
- Barrierefreiheit
- Responsive Design: gute Lesbarkeit auf Handy, Tablett und große Bildschirme
- Text soll vertikal zentriert und im Blickfeld des Betrachters sein
- Maßeinheiten in em
- Native Lazy Loading (für moderne Browser)
- Performance

Datei = " "

für index.html, html und php, css
\end{lstlisting}

\textbf{CSS optimieren}

\begin{lstlisting}
# ChatGPT
optimiere meine CSS und Vermeide teure CSS-Selektoren 

- Zusammenfassung von Wiederholungen
- Vermeidung von spezifischen Kombinationen:
- Optimierung der Media Queries: 

datei = " "
\end{lstlisting}

\textbf{Schreibstile}

\begin{lstlisting}
# ChatGPT
# Favorit: Expositorisch
# Schreibstile: Beschreibend, Expositorisch, Reflektierend, Akademisch, Kritisch
Schreibstil: Expositorisch
Erstellen Sie je nach Schreibstil eine kurze (ca. 35 Wörter) und ansprechende Zusammenfassung des folgenden Artikels. Die Zusammenfassung sollte für jemanden ohne wissenschaftlichen Hintergrund verständlich sein und gleichzeitig die wichtigsten Informationen genau wiedergeben. 
Artikel: " "
\end{lstlisting}

\hypertarget{schreibstile}{%
\section{Schreibstile}\label{schreibstile}}

\textbf{Beispiel 1}

\begin{lstlisting}
# ChatGPT
# Schreibstile: Beschreibend, Expositorisch, Reflektierend, Akademisch, Kritisch
Schreibstil: Expositorisch ohne Form du/sie
Erstellen Sie je nach Schreibstil eine kurze (ca. 35 Wörter) und ansprechende Zusammenfassung des folgenden Artikels. Die Zusammenfassung sollte für jemanden ohne wissenschaftlichen Hintergrund verständlich sein und gleichzeitig die wichtigsten Informationen genau wiedergeben. 
Artikel: " "
\end{lstlisting}

\textbf{Beispiel 2}

\begin{lstlisting}
# ChatGPT
# Schreibstil: Expositorisch ohne Form du/sie
Fassen Sie den folgenden Text als Aufzählung der wichtigsten Punkte zusammen. 
Text: " "
\end{lstlisting}

\hypertarget{programmiersprachen}{%
\section{Programmiersprachen}\label{programmiersprachen}}

Aufgaben zum C/C++ Kurs / Quelle: Universität Regensburg / Fakultät
Physik

\begin{lstlisting}
# ChatGPT
Aufgabe: "Schreiben Sie ein Programm, ..."

Entwickeln Sie ein Programm in den Programmiersprachen C, C++, Python, Arduino. 

Das Programm soll: Bei fehlerhaften Eingaben (z.B. wenn der Benutzer keine Zahl eingibt) 
sollte das Programm den Benutzer darauf hinweisen und ihn erneut zur Eingabe auffordern.

Anforderungen:

-   Implementieren Sie Fehlerprüfungen für die Eingabe.
-   Kommentieren Sie den Code auf Deutsch, wo es sinnvoll und notwendig ist.
-   Befolgen Sie die Best Practices

Nachdem Sie das Programm in allen fünf Programmiersprachen erstellt haben, 
dokumentieren Sie die Testergebnisse in Markdown.
\end{lstlisting}

\begin{lstlisting}
# ChatGPT
Nehmen Sie die Rolle eines erfahrenen Webprogrammierer und Webdesigner ein.
Aufgabe: " "

Entwickeln Sie ein Programm in der Programmiersprache PHP. 

Das Programm soll: Bei fehlerhaften Eingaben (z.B. wenn der Benutzer keine Zahl eingibt) 
sollte das Programm den Benutzer darauf hinweisen und ihn erneut zur Eingabe auffordern.

Anforderungen:

-   Implementieren Sie Fehlerprüfungen für die Eingabe.
-   Kommentieren Sie den Code auf Deutsch, wo es sinnvoll und notwendig ist.
-   Befolgen Sie die Best Practices
-  Sicherheit

Schreibstil: Expositorisch ohne Form du/sie
Textausgabe: in Markdown

Nachdem Sie das Programm erstellt haben, 
dokumentieren Sie die Testergebnisse und was macht das Programm.
\end{lstlisting}

\hypertarget{bachelor-arbeit}{%
\section{Bachelor-Arbeit}\label{bachelor-arbeit}}

\textbf{Beispiel 1}

\begin{lstlisting}
# ChatGPT
Nenne fünf globale Probleme der heutigen Zeit

Nenne mir fünf kritische Prüfungsfragen auf eine Bachelor-Arbeit zum fünften Thema
Schreibe mir eine Inhaltsangabe auf eine potenzielle Bachelor-Arbeit

# Alternative
Bitte schreib noch ein mögliches Inhaltsverzeichnis für solch eine Bachelor-Arbeit
\end{lstlisting}

\textbf{Beispiel 2}

\begin{lstlisting}
# ChatGPT
Du bist in der Rolle eines Hochschullehrers im Fach Kfz-Technik
Thema: Sensoren und Aktoren in der Fahrzeugtechnik
Nenne mir fünf kritische Prüfungsfragen auf eine Bachelor-Arbeit zum Thema

Schreibe mir eine Inhaltsangabe auf eine potenzielle Bachelor-Arbeit

# Alternative
Bitte schreib noch ein mögliches Inhaltsverzeichnis für solch eine Bachelor-Arbeit
\end{lstlisting}

\hypertarget{vortrag---90-minuxfctige-lektion-planen}{%
\section{Vortrag - 90-minütige Lektion
planen}\label{vortrag---90-minuxfctige-lektion-planen}}

\begin{lstlisting}
# ChatGPT
Du bist in der Rolle eines Hochschullehrers im Fach Kfz-Technik
Plane eine 90-minütige Lektion zum Thema: Sensoren
Lernziel ist Folgendes: Aufbau und Funktion von Sensoren erklären. 
Zielgruppe: Kfz-Meisterniveau
Zur Verfügung stehende Zeit: 90 Minuten.
Gib die Planung als Tabelle mit den Spalten: Termin, Inhalt, Methoden und notwendige Medien aus

Erstelle mir einen Vortrag zum 1. Thema

# Alternative
Bitte schreib noch eine mögliche PowerPoint-Präsentation zum 1. Thema
\end{lstlisting}

\hypertarget{erkluxe4re-das-thema}{%
\section{Erkläre das Thema}\label{erkluxe4re-das-thema}}

\begin{lstlisting}
# ChatGPT
Du bist in der Rolle eines Hochschullehrers im Fach Kfz-Technik
Zielgruppe: Kfz-Meisterniveau
Thema: 
keywords: 
Erkläre mir das Thema
# Erkläre mir, wie der Aktor vom Steuergerät angesteuert wird
# Erkläre mir, wie der Sensor das Signal an das Steuergerät sendet

Gib mir eine Zusammenfassung zu deiner Erklärung
# Alternative
Gib mir bitte nochmals eine verbesserte Zusammenfassung zu deiner Erklärung

Fasse mir die drei wichtigsten Punkte zusammen zu deiner Erklärung auf Meisterniveau
Gib mir eine kommagetrennte Liste mit den wichtigsten keywords zu deiner Erklärung

Erstelle mir bitte 10 Prüfungsfragen mit Lösungen zu deiner Erklärung
\end{lstlisting}

\hypertarget{lektionsplan-erstellen}{%
\section{Lektionsplan erstellen}\label{lektionsplan-erstellen}}

\begin{lstlisting}
# ChatGPT
Du bist in der Rolle eines Hochschullehrers im Fach Kfz-Technik
Erstelle mir einen Lektionsplan: Batterie in der Kfz-Technik (90 Minuten)
Lernziel: Aufbau und Funktion erklären. 
Zielgruppe: Kfz-Meisterniveau
keywords: 

# Erklären
Erkläre mir die Lektion 2

# Zusammenfassen
Fasse mir die drei wichtigsten Punkte zusammen zu deiner Erklärung auf Meisterniveau
# Keywords
Gib mir eine kommagetrennte Liste mit den wichtigsten keywords zu deiner Erklärung

# Fragen
Erstelle mir bitte 5 Prüfungsfragen mit Lösungen zu deiner Erklärung
\end{lstlisting}

\hypertarget{fachbuxfccher-zusammenfassen}{%
\section{Fachbücher zusammenfassen}\label{fachbuxfccher-zusammenfassen}}

\textbf{Text aus einem Fachbuch kopieren 1}

\begin{lstlisting}
# ChatGPT
Nehmen Sie die Rolle eines erfahrenen Hochschullehrers im Fach Kfz-Technik ein.
Zielgruppe: Kfz-Meisterniveau
# Text aus einem Fachbuch kopieren
Fassen Sie den folgenden Text als Aufzählung der wichtigsten Punkte zusammen. 
Text: " "

Erklären Sie einem Kfz-Meister das Thema anhand des Textes

Fasse mir die drei wichtigsten Punkte zu deiner Erklärung zusammen
Gib mir eine kommagetrennte Liste mit den wichtigsten keywords zu deiner Erklärung

Erstelle mir bitte 5 Prüfungsfragen mit Lösungen zu deiner Erklärung
\end{lstlisting}

\textbf{Text aus einem Fachbuch kopieren 2}

\begin{lstlisting}
# ChatGPT
Erstellen Sie eine kurze (ca. 200 Wörter) und ansprechende Zusammenfassung der folgenden wissenschaftlichen Arbeit. Die Zusammenfassung sollte für jemanden ohne wissenschaftlichen Hintergrund verständlich sein und gleichzeitig die wichtigsten Ergebnisse der Arbeit genau wiedergeben. 
Papier: " "
\end{lstlisting}

\hypertarget{prompt-engineering-techniken-und-best-practices}{%
\section{Prompt Engineering: Techniken und Best
Practices}\label{prompt-engineering-techniken-und-best-practices}}

\begin{enumerate}
\def\labelenumi{\arabic{enumi})}
\tightlist
\item
  Schnelle Platzierung und Beschreibung
\end{enumerate}

\begin{itemize}
\item
  {\lstinline!Erstellen Sie eine kurze (ca. 200 Wörter) und ansprechende Zusammenfassung der folgenden wissenschaftlichen Arbeit. Die Zusammenfassung sollte für jemanden ohne wissenschaftlichen Hintergrund verständlich sein und gleichzeitig die wichtigsten Ergebnisse der Arbeit genau wiedergeben. Papier: " "!}
\item
  {\lstinline!Fassen Sie den folgenden Text als Aufzählung der wichtigsten Punkte zusammen. Text: " "!}
\end{itemize}

\begin{enumerate}
\def\labelenumi{\arabic{enumi})}
\setcounter{enumi}{1}
\tightlist
\item
  Persona-Muster
\end{enumerate}

\begin{itemize}
\item
  {\lstinline!Nehmen Sie die Rolle eines erfahrenen Cybersicherheitsexperten an. Führen Sie anhand dieser Person ein Codeüberprüfung durch.!}
\item
  {\lstinline!Nehmen Sie die Rolle eines erfahrenen Historikers an, der sich auf die Französische Revolution spezialisiert hat. Erklären Sie anhand dieser Person die wichtigsten Ereignisse und Gründe, die zum Untergang der französischen Monarchie führten.!}
\end{itemize}

\textbf{Einführung neuer Informationen}

\begin{itemize}
\tightlist
\item
  {\lstinline!Erklären Sie einem Gymnasiasten, der sich mit digitalen Finanzen beschäftigt, das Konzept der Kryptowährung.!}
\end{itemize}

\begin{enumerate}
\def\labelenumi{\arabic{enumi})}
\setcounter{enumi}{2}
\tightlist
\item
  Größenbeschränkungen für Eingabeaufforderungen
\end{enumerate}

\begin{itemize}
\tightlist
\item
  {\lstinline!Angesichts der symbolischen Einschränkungen des Modells fassen Sie die wichtigsten Ereignisse des Zweiten Weltkriegs in weniger als 1000 Wörtern zusammen.!}
\end{itemize}

\begin{enumerate}
\def\labelenumi{\arabic{enumi})}
\setcounter{enumi}{3}
\tightlist
\item
  Fragenverfeinerungsmuster
\end{enumerate}

\begin{itemize}
\tightlist
\item
  {\lstinline!Wenn ich eine Frage zur Datenwissenschaft stelle, schlagen Sie eine verfeinerte Frage unter Berücksichtigung der Besonderheiten der statistischen Analyse vor und fragen Sie, ob ich mit der verfeinerten Frage fortfahren möchte.!}
\end{itemize}

\begin{enumerate}
\def\labelenumi{\arabic{enumi})}
\setcounter{enumi}{4}
\tightlist
\item
  Kognitives Prüfmuster
\end{enumerate}

\begin{itemize}
\tightlist
\item
  {\lstinline!Wenn ich eine Frage zum Klimawandel stelle, teilen Sie sie in drei kleinere Fragen auf, die Ihnen helfen, eine genauere Antwort zu geben. Kombinieren Sie die Antworten auf diese Unterfragen, um die endgültige Antwort zu erhalten.!}
\end{itemize}

\begin{enumerate}
\def\labelenumi{\arabic{enumi})}
\setcounter{enumi}{6}
\tightlist
\item
  Aufforderung zur Gedankenkette
\end{enumerate}

\begin{itemize}
\tightlist
\item
  {\lstinline!Könnten Sie kurz das Konzept der künstlichen Intelligenz erläutern?!}
\item
  {\lstinline!Wie beeinflusst KI den aktuellen Arbeitsmarkt?!}
\end{itemize} % Platzhalter

\end{document}
