% vorlage-main.tex
\documentclass[12pt,a4paper]{scrartcl}
\usepackage{vorlage-design-main}% vorlage-design-main.sty
% Bildgröße global ändern
\setkeys{Gin}{width=0.75\linewidth}
% Literatur
\addbibresource{referenzen.bib}
% Anpassung des Quellcode-Stils
\lstset{
  basicstyle=\ttfamily\small,   % Ändern Sie die Schriftgröße des Quellcodes, falls erforderlich.
  columns=fullflexible,
  breaklines=true,              % Zeilenumbrüche für zu lange Zeilen
  postbreak=\mbox{$\hookrightarrow$\space}, % Pfeil am Zeilenumbruch
  literate={ö}{{\"o}}1
           {ä}{{\"a}}1
           {ü}{{\"u}}1
           {Ö}{{\"O}}1
           {Ä}{{\"A}}1
           {Ü}{{\"U}}1
           {ß}{{\ss}}1,
  xleftmargin=2em,              % Optional: Linker Abstand
  xrightmargin=3em,             % Optional: Rechter Abstand
  showstringspaces=false,
  showspaces=false             % Zeigt keine Leerzeichen an
}
% Hyperlinks
\hypersetup{
    colorlinks=true,
    linkcolor=blue,
    filecolor=magenta,
    urlcolor=cyan,
}
% Fehler wenn pandoc - Markdown in Latex
\newcommand{\tightlist}{
  \setlength{\itemsep}{0pt}\setlength{\parskip}{0pt}
}
% Titel, Autor und Datum
\title{Mein optimiertes Dokument}
%\author{Jan Unger}
\date{\today}

\begin{document}
\maketitle

\hypertarget{markdown-zu-html-konverter}{%
\subsection{Markdown zu HTML
Konverter}\label{markdown-zu-html-konverter}}

\hypertarget{uxfcberblick}{%
\subsubsection{Überblick}\label{uxfcberblick}}

Dieses Skript ermöglicht die Konvertierung von Text, der in Markdown
verfasst wurde, in HTML. Es ist in PHP geschrieben und verwendet die
{\lstinline!Parsedown!}-Bibliothek, um eine korrekte
Umsetzung des Markdown-Formats sicherzustellen.

\hypertarget{installation}{%
\subsubsection{Installation}\label{installation}}

\begin{enumerate}
\def\labelenumi{\arabic{enumi}.}
\item
  \textbf{Voraussetzungen}

  \begin{itemize}
  \tightlist
  \item
    Ein Server mit PHP-Unterstützung.
  \item
    \href{https://getcomposer.org/}{Composer}, ein Werkzeug für die
    Abhängigkeitsverwaltung in PHP.
  \end{itemize}
\item
  \textbf{Bibliothek installieren} Nach der Installation von Composer
  führt man den folgenden Befehl im Hauptverzeichnis des Projekts aus:

\begin{lstlisting}[language=bash]
# Terminal
php -r "copy('https://getcomposer.org/installer', 'composer-setup.php');"
mkdir -p /usr/local/bin
mv composer.phar /usr/local/bin/composer
composer
composer require erusev/parsedown
\end{lstlisting}

  Dies wird die {\lstinline!Parsedown!}-Bibliothek in das
  Projekt herunterladen und installieren.
\item
  \textbf{Skript bereitstellen} Das zuvor bereitgestellte PHP-Skript
  kann in einem geeigneten Verzeichnis auf dem Server abgelegt werden.
\end{enumerate}

\hypertarget{scriptaufruf}{%
\subsubsection{Scriptaufruf}\label{scriptaufruf}}

Nachdem alles installiert wurde, kann man das Skript über einen
Webbrowser aufrufen, indem man die entsprechende URL in die Adressleiste
des Browsers eingibt. Zum Beispiel:

\begin{lstlisting}
http://meinedomain.de/pfad/zum/script.php
\end{lstlisting}

\hypertarget{funktionsweise}{%
\subsubsection{Funktionsweise}\label{funktionsweise}}

Nach dem Aufrufen des Skripts im Browser sieht man ein Formularfeld, in
das Markdown-Text eingegeben werden kann. Nachdem der Text eingegeben
und das Formular abgeschickt wurde, zeigt das Skript den eingegebenen
Markdown-Text als HTML formatiert an.

\hypertarget{testergebnisse}{%
\subsubsection{Testergebnisse}\label{testergebnisse}}

Das Skript wurde mit verschiedenen Markdown-Eingaben getestet,
einschließlich, aber nicht beschränkt auf:

\begin{itemize}
\tightlist
\item
  Überschriften
\item
  Listen (sowohl geordnet als auch ungeordnet)
\item
  Links
\item
  Bilder
\item
  Textformatierungen wie \textbf{Fett}, \emph{Kursiv} und
  {\lstinline!Code!}.
\end{itemize}

In allen Testszenarien wurde der Markdown-Text korrekt in HTML
umgewandelt und angezeigt.

\hypertarget{sicherheitshinweis}{%
\subsubsection{Sicherheitshinweis}\label{sicherheitshinweis}}

Das Skript verwendet {\lstinline!htmlspecialchars!} zur
Behandlung von Benutzereingaben, um Cross-Site-Scripting-Attacken (XSS)
zu verhindern. Es ist jedoch immer ratsam, weitere sicherheitsrelevante
Aspekte im Auge zu behalten und das Skript regelmäßig auf
Sicherheitslücken zu überprüfen, insbesondere wenn es in einer
Produktionsumgebung eingesetzt wird.

\hypertarget{schlussfolgerung}{%
\subsubsection{Schlussfolgerung}\label{schlussfolgerung}}

Das bereitgestellte Skript ist ein nützliches Werkzeug, um Markdown
schnell und einfach in HTML umzuwandeln. Es bietet eine einfache
Benutzeroberfläche und zuverlässige
Konvertierungsfunktionen. % Platzhalter

\end{document}
