% vorlage-main.tex
\documentclass[12pt,a4paper]{scrartcl}
\usepackage{vorlage-design-main}% vorlage-design-main.sty
% Bildgröße global ändern
\setkeys{Gin}{width=0.75\linewidth}
% Literatur
\addbibresource{referenzen.bib}
% Anpassung des Quellcode-Stils
\lstset{
  basicstyle=\ttfamily\small,   % Ändern Sie die Schriftgröße des Quellcodes, falls erforderlich.
  columns=fullflexible,
  breaklines=true,              % Zeilenumbrüche für zu lange Zeilen
  postbreak=\mbox{$\hookrightarrow$\space}, % Pfeil am Zeilenumbruch
  literate={ö}{{\"o}}1
           {ä}{{\"a}}1
           {ü}{{\"u}}1
           {Ö}{{\"O}}1
           {Ä}{{\"A}}1
           {Ü}{{\"U}}1
           {ß}{{\ss}}1,
  xleftmargin=2em,              % Optional: Linker Abstand
  xrightmargin=3em,             % Optional: Rechter Abstand
  showstringspaces=false,
  showspaces=false             % Zeigt keine Leerzeichen an
}
% Hyperlinks
\hypersetup{
    colorlinks=true,
    linkcolor=blue,
    filecolor=magenta,
    urlcolor=cyan,
}
% Fehler wenn pandoc - Markdown in Latex
\newcommand{\tightlist}{
  \setlength{\itemsep}{0pt}\setlength{\parskip}{0pt}
}
% Titel, Autor und Datum
\title{Mein optimiertes Dokument}
%\author{Jan Unger}
\date{\today}

\begin{document}
\maketitle

\hypertarget{parallelschaltung}{%
\section{Parallelschaltung}\label{parallelschaltung}}

\textbf{Aufgabenstellung: Elektrische Eigenschaften einer
Parallelschaltung von Widerständen}

\textbf{Teil A:} Entwickle ein Python-Programm, das die elektrischen
Eigenschaften einer Parallelschaltung von drei Widerständen berechnet.

\textbf{Eingabe:} Drei Widerstandswerte in Ohm und eine Spannungsquelle
von 12 V.

\textbf{Gesuchte Ausgaben:}

\begin{itemize}
\tightlist
\item
  Teilströme über jeden Widerstand.
\item
  Gesamtstrom durch die Schaltung.
\item
  Gesamtleistung der Schaltung.
\item
  Leistung über jeden Widerstand.
\item
  Gesamtwiderstand der Schaltung.
\end{itemize}

Nutze das Modul {\lstinline!matplotlib!} für Python, um die
Ergebnisse grafisch darzustellen. Teste dein Programm anschließend mit
drei exemplarischen Widerstandswerten.

\textbf{Teil B:} Erläutere die Berechnungen für eine Parallelschaltung
von Widerständen mit Hilfe von Markdown.

\textbf{Inhalt:}

\begin{itemize}
\tightlist
\item
  Beschreibung der gegebenen Werte und der gesuchten Größen.
\item
  Manuelle Berechnung der oben genannten Größen.
\item
  Darstellung der Parallelschaltung mit Hilfe eines einfachen
  ASCII-Schaltplans.
\end{itemize}

Verwende die gegebenen Formeln und Lösungen, um die Teile A und B
entsprechend auszuarbeiten.

\textbf{Schaltplan}

\begin{lstlisting}
       +--R1--+
       |      |
+12V---+--R2--+---GND
       |      |
       +--R3--+
\end{lstlisting}

\newpage

\textbf{Python-Code zur Berechnung und grafischen Darstellung}

\begin{lstlisting}[language=Python]
# Quellcode in Python
import matplotlib.pyplot as plt
import matplotlib 
matplotlib.rcParams['text.usetex'] = True # Latex code

# Funktion zur Berechnung der Werte für eine Parallelschaltung
def parallelschaltung(R1, R2, R3, U):
    # Berechnung des Gesamtwiderstands
    R_ges = 1 / (1/R1 + 1/R2 + 1/R3)
    # Teilströme berechnen
    I1 = U / R1
    I2 = U / R2
    I3 = U / R3
    # Gesamtstrom
    I_ges = I1 + I2 + I3
    # Leistungen berechnen
    P_ges = U * I_ges
    P1 = U * I1
    P2 = U * I2
    P3 = U * I3
    return I1, I2, I3, I_ges, P_ges, P1, P2, P3, R_ges

# Testwerte
R1 = 10  # Ohm
R2 = 20  # Ohm
R3 = 30  # Ohm
U = 12   # Volt

I1, I2, I3, I_ges, P_ges, P1, P2, P3, R_ges = parallelschaltung(R1, R2, R3, U)

# Ausgabe der berechneten Werte
print(f"Teilstrom über R1, I1 = {I1:.2f} A")
print(f"Teilstrom über R2, I2 = {I2:.2f} A")
print(f"Teilstrom über R3, I3 = {I3:.2f} A")
print(f"Gesamtstrom, I_ges = {I_ges:.2f} A")
print(f"Gesamtleistung, P_ges = {P_ges:.2f} W")
print(f"Leistung über R1, P1 = {P1:.2f} W")
print(f"Leistung über R2, P2 = {P2:.2f} W")
print(f"Leistung über R3, P3 = {P3:.2f} W")
print(f"Gesamtwiderstand, R_ges = {R_ges:.2f} Ohm")

# Grafische Darstellung
widerstaende = ["R1", "R2", "R3"]
teilstroeme = [I1, I2, I3]
leistungen = [P1, P2, P3]

# Farbe: Orange #F28C64 grau2 #B2B2B2 rot5 #A71916 blau5 #0D468E
plt.figure(figsize=(250/25.4, 176/25.4))  # Größe in inches (B5 format: 250mm x 176mm)plt.figure(figsize=(12, 5))
plt.subplot(1, 2, 1)
plt.bar(widerstaende, teilstroeme, color='#0D468E')#blau5 #0D468E
plt.title(r'Teilströme',fontsize=12)#
plt.ylabel(r'\textbf{Strom (A)}')
plt.xlabel(r'\textbf{Widerstand}')
plt.subplot(1, 2, 2)
plt.bar(widerstaende, leistungen, color='#A71916')#rot5 #A71916
plt.title(r'Leistungen',fontsize=12)#
plt.ylabel(r'\textbf{Leistung (W)}')
plt.xlabel(r'\textbf{Widerstand}')
plt.tight_layout()
plt.savefig("Diag_Parallelschaltung.svg")# SVG-Vektorgrafik
plt.show()
\end{lstlisting}

\textbf{Programmberechnung}

\begin{lstlisting}[language=Python]
# Ausgabe Quellcode:
Teilstrom über R1, I1 = 1.20 A
Teilstrom über R2, I2 = 0.60 A
Teilstrom über R3, I3 = 0.40 A
Gesamtstrom, I_ges = 2.20 A
Gesamtleistung, P_ges = 26.40 W
Leistung über R1, P1 = 14.40 W
Leistung über R2, P2 = 7.20 W
Leistung über R3, P3 = 4.80 W
Gesamtwiderstand, R_ges = 5.45 Ohm
\end{lstlisting}

\includesvg[width=0.5\textwidth,height=\textheight]{../images/Diag_Parallelschaltung-min.svg}

\newpage

\textbf{Berechnung}

\textbf{Gegebene Werte:}

\begin{itemize}
\tightlist
\item
  \(R_1 = 10~\Omega\)
\item
  \(R_2 = 20~\Omega\)
\item
  \(R_3 = 30~\Omega\)
\item
  \(U = 12~V\)
\end{itemize}

\textbf{Gesamtwiderstand der Parallelschaltung:}

\begin{itemize}
\item
  \(\frac{1}{R_{ges}} = \frac{1}{R_1} + \frac{1}{R_2} + \frac{1}{R_3} = \frac{1}{10} + \frac{1}{20} + \frac{1}{30} = 0.1833\)
\item
  Daraus ergibt sich: \(R_{ges} = \frac{1}{0.1833} = 5.46~\Omega\)
\end{itemize}

\textbf{Teilströme:}

\begin{itemize}
\tightlist
\item
  \(I_1 = \frac{U}{R_1} = \frac{12}{10} = 1.2~A\)
\item
  \(I_2 = \frac{U}{R_2} = \frac{12}{20} = 0.6~A\)
\item
  \(I_3 = \frac{U}{R_3} = \frac{12}{30} = 0.4~A\)
\end{itemize}

\textbf{Gesamtstrom:}

\begin{itemize}
\tightlist
\item
  \(I_{ges} = I_1 + I_2 + I_3 = 1.2 + 0.6 + 0.4 = 2.2~A\)
\end{itemize}

\textbf{Leistung über jeden Widerstand:}

\begin{itemize}
\tightlist
\item
  \(P_1 = U \times I_1 = 12 \times 1.2 = 14.4~W\)
\item
  \(P_2 = U \times I_2 = 12 \times 0.6 = 7.2~W\)
\item
  \(P_3 = U \times I_3 = 12 \times 0.4 = 4.8~W\)
\end{itemize}

\textbf{Gesamtleistung:}

\begin{itemize}
\tightlist
\item
  \(P_{ges} = U \times I_{ges} = 12 \times 2.2 = 26.4~W\)
\end{itemize}

\textbf{Ergebnisse:}

\begin{itemize}
\tightlist
\item
  Gesamtwiderstand, \(R_{ges} = 5.46~\Omega\)
\item
  Teilstrom über \(R_1, I_1 = 1.2~A\)
\item
  Teilstrom über \(R_2, I_2 = 0.6~A\)
\item
  Teilstrom über \(R_3, I_3 = 0.4~A\)
\item
  Gesamtstrom, \(I_{ges} = 2.2~A\)
\item
  Leistung über \(R_1, P_1 = 14.4~W\)
\item
  Leistung über \(R_2, P_2 = 7.2~W\)
\item
  Leistung über \(R_3, P_3 = 4.8~W\)
\item
  Gesamtleistung, \(P_{ges} = 26.4~W\)
\end{itemize} % Platzhalter

\end{document}
